%//////////////////////////////////////////////////////////////////////////////
%
% Copyright (c) 2007,2009 Daniel Adler <dadler@uni-goettingen.de>, 
%                         Tassilo Philipp <tphilipp@potion-studios.com>
%
% Permission to use, copy, modify, and distribute this software for any
% purpose with or without fee is hereby granted, provided that the above
% copyright notice and this permission notice appear in all copies.
%
% THE SOFTWARE IS PROVIDED "AS IS" AND THE AUTHOR DISCLAIMS ALL WARRANTIES
% WITH REGARD TO THIS SOFTWARE INCLUDING ALL IMPLIED WARRANTIES OF
% MERCHANTABILITY AND FITNESS. IN NO EVENT SHALL THE AUTHOR BE LIABLE FOR
% ANY SPECIAL, DIRECT, INDIRECT, OR CONSEQUENTIAL DAMAGES OR ANY DAMAGES
% WHATSOEVER RESULTING FROM LOSS OF USE, DATA OR PROFITS, WHETHER IN AN
% ACTION OF CONTRACT, NEGLIGENCE OR OTHER TORTIOUS ACTION, ARISING OUT OF
% OR IN CONNECTION WITH THE USE OR PERFORMANCE OF THIS SOFTWARE.
%
%//////////////////////////////////////////////////////////////////////////////

% ==================================================
% x86 
% ==================================================
\subsection{x86 Calling Conventions}


\paragraph{Overview}

There are numerous different calling conventions on the x86 processor
architecture, like cdecl \cite{x86cdecl}, MS fastcall \cite{x86Winfastcall}, GNU
fastcall \cite{x86GNUfastcall}, Borland fastcall \cite{x86Borlandfastcall}, Watcom
fastcall \cite{x86Watcomfastcall}, Win32 stdcall \cite{x86Winstdcall}, MS thiscall
\cite{x86Winthiscall}, GNU thiscall \cite{x86GNUthiscall}, the pascal calling
convention \cite{x86Pascal} and a cdecl-like version for Plan9 \cite{x86Plan9}
(dubbed plan9call by us), etc.


\paragraph{\product{dyncall} support}

Currently cdecl, stdcall, fastcall (MS and GNU), thiscall (MS and GNU) and
plan9call are supported.\\
\\


\subsubsection{cdecl}

\paragraph{Registers and register usage}

\begin{table}[h]
\begin{tabular}{3 B}
\hline
Name          & Brief description\\
\hline
{\bf eax}     & scratch, return value\\
{\bf ebx}     & permanent\\
{\bf ecx}     & scratch\\
{\bf edx}     & scratch, return value\\
{\bf esi}     & permanent\\
{\bf edi}     & permanent\\
{\bf ebp}     & permanent\\
{\bf esp}     & stack pointer\\
{\bf st0}     & scratch, floating point return value\\
{\bf st1-st7} & scratch\\
\hline
\end{tabular}
\caption{Register usage on x86 cdecl calling convention}
\end{table}


\pagebreak

\paragraph{Parameter passing}

\begin{itemize}
\item stack parameter order: right-to-left
\item caller cleans up the stack
\item all arguments are pushed onto the stack
\end{itemize}

\paragraph{Return values}

\begin{itemize}
\item return values of pointer or integral type (\textless=\ 32 bits) are returned via the eax register
\item integers \textgreater\ 32 bits are returned via the eax and edx registers
\item floating point types are returned via the st0 register
\end{itemize}


\paragraph{Stack layout}

Stack directly after function prolog:\\

\begin{figure}[h]
\begin{tabular}{5|3|1 1}
\hhline{~-~~}
                                  & \vdots                     &                                &                              \\
\hhline{~=~~}
local data                        &                            &                                & \mrrbrace{5}{caller's frame} \\
\hhline{~-~~}
\mrlbrace{3}{parameter area}      & \ldots                     & \mrrbrace{3}{stack parameters} &                              \\
                                  & \ldots                     &                                &                              \\
                                  & \ldots                     &                                &                              \\
\hhline{~-~~}
                                  & return address             &                                &                              \\
\hhline{~=~~}
local data                        &                            &                                & \mrrbrace{3}{current frame}  \\
\hhline{~-~~}
parameter area                    &                            &                                &                              \\
\hhline{~-~~}
                                  & \vdots                     &                                &                              \\
\hhline{~-~~}
\end{tabular}
\caption{Stack layout on x86 cdecl calling convention}
\end{figure}


\pagebreak

\subsubsection{MS fastcall}

\paragraph{Registers and register usage}

\begin{table}[h]
\begin{tabular}{3 B}
\hline
Name          & Brief description\\
\hline
{\bf eax}     & scratch, return value\\
{\bf ebx}     & permanent\\
{\bf ecx}     & scratch, parameter 0\\
{\bf edx}     & scratch, parameter 1, return value\\
{\bf esi}     & permanent\\
{\bf edi}     & permanent\\
{\bf ebp}     & permanent\\
{\bf esp}     & stack pointer\\
{\bf st0}     & scratch, floating point return value\\
{\bf st1-st7} & scratch\\
\hline
\end{tabular}
\caption{Register usage on x86 fastcall (MS) calling convention}
\end{table}


\pagebreak

\paragraph{Parameter passing}

\begin{itemize}
\item stack parameter order: right-to-left
\item called function cleans up the stack
\item first two integers/pointers (\textless=\ 32bit) are passed via ecx and edx (even if preceded by other arguments)
\item integer types 64 bits in size @@@ ? first in edx:eax ?
\item if first argument is a 64 bit integer, it is passed via ecx and edx
\item all other parameters are pushed onto the stack
\end{itemize}

\paragraph{Return values}

\begin{itemize}
\item return values of pointer or integral type (\textless=\ 32 bits) are returned via the eax register
\item integers \textgreater\ 32 bits are returned via the eax and edx registers@@@verify
\item floating point types are returned via the st0 register@@@ really ?
\end{itemize}


\paragraph{Stack layout}

Stack directly after function prolog:\\

\begin{figure}[h]
\begin{tabular}{5|3|1 1}
\hhline{~-~~}
                                  & \vdots                     &                                &                              \\
\hhline{~=~~}
local data                        &                            &                                & \mrrbrace{5}{caller's frame} \\
\hhline{~-~~}
\mrlbrace{3}{parameter area}      & \ldots                     & \mrrbrace{3}{stack parameters} &                              \\
                                  & \ldots                     &                                &                              \\
                                  & \ldots                     &                                &                              \\
\hhline{~-~~}
                                  & return address             &                                &                              \\
\hhline{~=~~}
local data                        &                            &                                & \mrrbrace{3}{current frame}  \\
\hhline{~-~~}
parameter area                    &                            &                                &                              \\
\hhline{~-~~}
                                  & \vdots                     &                                &                              \\
\hhline{~-~~}
\end{tabular}
\caption{Stack layout on x86 fastcall (MS) calling convention}
\end{figure}


\pagebreak

\subsubsection{GNU fastcall}

\paragraph{Registers and register usage}

\begin{table}[h]
\begin{tabular}{3 B}
\hline
Name          & Brief description\\
\hline
{\bf eax}     & scratch, return value\\
{\bf ebx}     & permanent\\
{\bf ecx}     & scratch, parameter 0\\
{\bf edx}     & scratch, parameter 1, return value\\
{\bf esi}     & permanent\\
{\bf edi}     & permanent\\
{\bf ebp}     & permanent\\
{\bf esp}     & stack pointer\\
{\bf st0}     & scratch, floating point return value\\
{\bf st1-st7} & scratch\\
\hline
\end{tabular}
\caption{Register usage on x86 fastcall (GNU) calling convention}
\end{table}

\paragraph{Parameter passing}

\begin{itemize}
\item stack parameter order: right-to-left
\item called function cleans up the stack
\item first two integers/pointers (\textless=\ 32bit) are passed via ecx and edx (even if preceded by other arguments)
\item if first argument is a 64 bit integer, it is pushed on the stack and the two registers are skipped 
\item all other parameters are pushed onto the stack
\end{itemize}


\paragraph{Return values}

\begin{itemize}
\item return values of pointer or integral type (\textless=\ 32 bits) are returned via the eax register.
\item integers \textgreater\ 32 bits are returned via the eax and edx registers.
\item floating point types are returned via the st0.
\end{itemize}


\pagebreak

\paragraph{Stack layout}

Stack directly after function prolog:\\

\begin{figure}[h]
\begin{tabular}{5|3|1 1}
\hhline{~-~~}
                                  & \vdots                     &                                &                              \\
\hhline{~=~~}
local data                        &                            &                                & \mrrbrace{5}{caller's frame} \\
\hhline{~-~~}
\mrlbrace{3}{parameter area}      & \ldots                     & \mrrbrace{3}{stack parameters} &                              \\
                                  & \ldots                     &                                &                              \\
                                  & \ldots                     &                                &                              \\
\hhline{~-~~}
                                  & return address             &                                &                              \\
\hhline{~=~~}
local data                        &                            &                                & \mrrbrace{3}{current frame}  \\
\hhline{~-~~}
parameter area                    &                            &                                &                              \\
\hhline{~-~~}
                                  & \vdots                     &                                &                              \\
\hhline{~-~~}
\end{tabular}
\caption{Stack layout on x86 fastcall (GNU) calling convention}
\end{figure}


\subsubsection{Borland fastcall}

\paragraph{Registers and register usage}

\begin{table}[h]
\begin{tabular}{3 B}
\hline
Name          & Brief description\\
\hline
{\bf eax}     & scratch, parameter 0, return value\\
{\bf ebx}     & permanent\\
{\bf ecx}     & scratch, parameter 2\\
{\bf edx}     & scratch, parameter 1, return value\\
{\bf esi}     & permanent\\
{\bf edi}     & permanent\\
{\bf ebp}     & permanent\\
{\bf esp}     & stack pointer\\
{\bf st0}     & scratch, floating point return value\\
{\bf st1-st7} & scratch\\
\hline
\end{tabular}
\caption{Register usage on x86 fastcall (Borland) calling convention}
\end{table}

\paragraph{Parameter passing}

\begin{itemize}
\item stack parameter order: left-to-right
\item called function cleans up the stack
\item first three integers/pointers (\textless=\ 32bit) are passed via eax, ecx and edx (even if preceded by other arguments@@@?)
\item integer types 64 bits in size @@@ ?
\item all other parameters are pushed onto the stack
\end{itemize}


\pagebreak

\paragraph{Return values}

\begin{itemize}
\item return values of pointer or integral type (\textless=\ 32 bits) are returned via the eax register
\item integers \textgreater\ 32 bits are returned via the eax and edx registers@@@ verify
\item floating point types are returned via the st0 register@@@ really ?
\end{itemize}



\paragraph{Stack layout}

Stack directly after function prolog:\\

\begin{figure}[h]
\begin{tabular}{5|3|1 1}
\hhline{~-~~}
                                  & \vdots                     &                                &                              \\
\hhline{~=~~}
local data                        &                            &                                & \mrrbrace{5}{caller's frame} \\
\hhline{~-~~}
\mrlbrace{3}{parameter area}      & \ldots                     & \mrrbrace{3}{stack parameters} &                              \\
                                  & \ldots                     &                                &                              \\
                                  & \ldots                     &                                &                              \\
\hhline{~-~~}
                                  & return address             &                                &                              \\
\hhline{~=~~}
local data                        &                            &                                & \mrrbrace{3}{current frame}  \\
\hhline{~-~~}
parameter area                    &                            &                                &                              \\
\hhline{~-~~}
                                  & \vdots                     &                                &                              \\
\hhline{~-~~}
\end{tabular}
\caption{Stack layout on x86 fastcall (Borland) calling convention}
\end{figure}


\subsubsection{Watcom fastcall}


\paragraph{Registers and register usage}

\begin{table}[h]
\begin{tabular}{3 B}
\hline
Name          & Brief description\\
\hline
{\bf eax}     & scratch, parameter 0, return value@@@\\
{\bf ebx}     & scratch when used for parameter, parameter 2\\
{\bf ecx}     & scratch when used for parameter, parameter 3\\
{\bf edx}     & scratch when used for parameter, parameter 1, return value@@@\\
{\bf esi}     & scratch when used for return pointer @@@??\\
{\bf edi}     & permanent\\
{\bf ebp}     & permanent\\
{\bf esp}     & stack pointer\\
{\bf st0}     & scratch, floating point return value\\
{\bf st1-st7} & scratch\\
\hline
\end{tabular}
\caption{Register usage on x86 fastcall (Watcom) calling convention}
\end{table}

\paragraph{Parameter passing}

\begin{itemize}
\item stack parameter order: right-to-left
\item called function cleans up the stack
\item first four integers/pointers (\textless=\ 32bit) are passed via eax, edx, ebx and ecx (even if preceded by other arguments@@@?)
\item integer types 64 bits in size @@@ ?
\item all other parameters are pushed onto the stack
\end{itemize}


\paragraph{Return values}

\begin{itemize}
\item return values of pointer or integral type (\textless=\ 32 bits) are returned via the eax register@@@verify, I thnik its esi?
\item integers \textgreater\ 32 bits are returned via the eax and edx registers@@@ verify
\item floating point types are returned via the st0 register@@@ really ?
\end{itemize}


\paragraph{Stack layout}

Stack directly after function prolog:\\

\begin{figure}[h]
\begin{tabular}{5|3|1 1}
\hhline{~-~~}
                                  & \vdots                     &                                &                              \\
\hhline{~=~~}
local data                        &                            &                                & \mrrbrace{5}{caller's frame} \\
\hhline{~-~~}
\mrlbrace{3}{parameter area}      & \ldots                     & \mrrbrace{3}{stack parameters} &                              \\
                                  & \ldots                     &                                &                              \\
                                  & \ldots                     &                                &                              \\
\hhline{~-~~}
                                  & return address             &                                &                              \\
\hhline{~=~~}
local data                        &                            &                                & \mrrbrace{3}{current frame}  \\
\hhline{~-~~}
parameter area                    &                            &                                &                              \\
\hhline{~-~~}
                                  & \vdots                     &                                &                              \\
\hhline{~-~~}
\end{tabular}
\caption{Stack layout on x86 fastcall (Watcom) calling convention}
\end{figure}



\subsubsection{win32 stdcall}

\paragraph{Registers and register usage}

\begin{table}[h]
\begin{tabular}{3 B}
\hline
Name          & Brief description\\
\hline
{\bf eax}     & scratch, return value\\
{\bf ebx}     & permanent\\
{\bf ecx}     & scratch\\
{\bf edx}     & scratch, return value\\
{\bf esi}     & permanent\\
{\bf edi}     & permanent\\
{\bf ebp}     & permanent\\
{\bf esp}     & stack pointer\\
{\bf st0}     & scratch, floating point return value\\
{\bf st1-st7} & scratch\\
\hline
\end{tabular}
\caption{Register usage on x86 stdcall calling convention}
\end{table}

\paragraph{Parameter passing}

\begin{itemize}
\item Stack parameter order: right-to-left
\item Called function cleans up the stack
\item All parameters are pushed onto the stack
\item Stack is usually 4 byte aligned (GCC \textgreater=\ 3.x seems to use a 16byte alignement@@@)
\item Function name is decorated by prepending an underscore character and appending a '@' character and the number of bytes of stack space required
\end{itemize}


\paragraph{Return values}

\begin{itemize}
\item return values of pointer or integral type (\textless=\ 32 bits) are returned via the eax register
\item integers \textgreater\ 32 bits are returned via the eax and edx registers
\item floating point types are returned via the st0 register
\end{itemize}


\paragraph{Stack layout}

Stack directly after function prolog:\\

\begin{figure}[h]
\begin{tabular}{5|3|1 1}
\hhline{~-~~}
                                  & \vdots                     &                                &                              \\
\hhline{~=~~}
local data                        &                            &                                & \mrrbrace{5}{caller's frame} \\
\hhline{~-~~}
\mrlbrace{3}{parameter area}      & \ldots                     & \mrrbrace{3}{stack parameters} &                              \\
                                  & \ldots                     &                                &                              \\
                                  & \ldots                     &                                &                              \\
\hhline{~-~~}
                                  & return address             &                                &                              \\
\hhline{~=~~}
local data                        &                            &                                & \mrrbrace{3}{current frame}  \\
\hhline{~-~~}
parameter area                    &                            &                                &                              \\
\hhline{~-~~}
                                  & \vdots                     &                                &                              \\
\hhline{~-~~}
\end{tabular}
\caption{Stack layout on x86 stdcall calling convention}
\end{figure}

\subsubsection{MS thiscall}

\paragraph{Registers and register usage}

\begin{table}[h]
\begin{tabular}{3 B}
\hline
Name          & Brief description\\
\hline
{\bf eax}     & scratch, return value\\
{\bf ebx}     & permanent\\
{\bf ecx}     & scratch, parameter 0\\
{\bf edx}     & scratch, return value\\
{\bf esi}     & permanent\\
{\bf edi}     & permanent\\
{\bf ebp}     & permanent\\
{\bf esp}     & stack pointer\\
{\bf st0}     & scratch, floating point return value\\
{\bf st1-st7} & scratch\\
\hline
\end{tabular}
\caption{Register usage on x86 thiscall (MS) calling convention}
\end{table}

\newpage


\paragraph{Parameter passing}

\begin{itemize}
\item stack parameter order: right-to-left
\item called function cleans up the stack
\item first parameter (this pointer) is passed via ecx
\item all other parameters are pushed onto the stack
\item Function name is decorated by prepending a '@' character and appending a '@' character and the number of bytes (decimal) of stack space required
\end{itemize}


\paragraph{Return values}

\begin{itemize}
\item return values of pointer or integral type (\textless=\ 32 bits) are returned via the eax register
\item integers \textgreater\ 32 bits are returned via the eax and edx registers@@@verify
\item floating point types are returned via the st0 register@@@ really ?
\end{itemize}


\paragraph{Stack layout}

Stack directly after function prolog:\\

\begin{figure}[h]
\begin{tabular}{5|3|1 1}
\hhline{~-~~}
                                  & \vdots                     &                                &                              \\
\hhline{~=~~}
local data                        &                            &                                & \mrrbrace{5}{caller's frame} \\
\hhline{~-~~}
\mrlbrace{3}{parameter area}      & \ldots                     & \mrrbrace{3}{stack parameters} &                              \\
                                  & \ldots                     &                                &                              \\
                                  & \ldots                     &                                &                              \\
\hhline{~-~~}
                                  & return address             &                                &                              \\
\hhline{~=~~}
local data                        &                            &                                & \mrrbrace{3}{current frame}  \\
\hhline{~-~~}
parameter area                    &                            &                                &                              \\
\hhline{~-~~}
                                  & \vdots                     &                                &                              \\
\hhline{~-~~}
\end{tabular}
\caption{Stack layout on x86 thiscall (MS) calling convention}
\end{figure}



\subsubsection{GNU thiscall}

\paragraph{Registers and register usage}

\begin{table}[h]
\begin{tabular}{3 B}
\hline
Name          & Brief description\\
\hline
{\bf eax}     & scratch, return value\\
{\bf ebx}     & permanent\\
{\bf ecx}     & scratch\\
{\bf edx}     & scratch, return value\\
{\bf esi}     & permanent\\
{\bf edi}     & permanent\\
{\bf ebp}     & permanent\\
{\bf esp}     & stack pointer\\
{\bf st0}     & scratch, floating point return value\\
{\bf st1-st7} & scratch\\
\hline
\end{tabular}
\caption{Register usage on x86 thiscall (GNU) calling convention}
\end{table}

\paragraph{Parameter passing}

\begin{itemize}
\item stack parameter order: right-to-left
\item caller cleans up the stack
\item all parameters are pushed onto the stack
\end{itemize}


\paragraph{Return values}

\begin{itemize}
\item return values of pointer or integral type (\textless=\ 32 bits) are returned via the eax register
\item integers \textgreater\ 32 bits are returned via the eax and edx registers@@@verify
\item floating point types are returned via the st0 register@@@ really ?
\end{itemize}


\paragraph{Stack layout}

Stack directly after function prolog:\\

\begin{figure}[h]
\begin{tabular}{5|3|1 1}
\hhline{~-~~}
                                  & \vdots                     &                                &                              \\
\hhline{~=~~}
local data                        &                            &                                & \mrrbrace{5}{caller's frame} \\
\hhline{~-~~}
\mrlbrace{3}{parameter area}      & \ldots                     & \mrrbrace{3}{stack parameters} &                              \\
                                  & \ldots                     &                                &                              \\
                                  & \ldots                     &                                &                              \\
\hhline{~-~~}
                                  & return address             &                                &                              \\
\hhline{~=~~}
local data                        &                            &                                & \mrrbrace{3}{current frame}  \\
\hhline{~-~~}
parameter area                    &                            &                                &                              \\
\hhline{~-~~}
                                  & \vdots                     &                                &                              \\
\hhline{~-~~}
\end{tabular}
\caption{Stack layout on x86 thiscall (GNU) calling convention}
\end{figure}



\subsubsection{pascal}

The best known uses of the pascal calling convention are the 16 bit OS/2 APIs, Microsoft Windows 3.x and Borland Delphi 1.x.

\paragraph{Registers and register usage}

\begin{table}[h]
\begin{tabular}{3 B}
\hline
Name          & Brief description\\
\hline
{\bf eax}     & scratch, return value\\
{\bf ebx}     & permanent\\
{\bf ecx}     & scratch\\
{\bf edx}     & scratch, return value\\
{\bf esi}     & permanent\\
{\bf edi}     & permanent\\
{\bf ebp}     & permanent\\
{\bf esp}     & stack pointer\\
{\bf st0}     & scratch, floating point return value\\
{\bf st1-st7} & scratch\\
\hline
\end{tabular}
\caption{Register usage on x86 pascal calling convention}
\end{table}

\paragraph{Parameter passing}

\begin{itemize}
\item stack parameter order: left-to-right
\item called function cleans up the stack
\item all parameters are pushed onto the stack
\end{itemize}


\paragraph{Return values}

\begin{itemize}
\item return values of pointer or integral type (\textless=\ 32 bits) are returned via the eax register
\item integers \textgreater\ 32 bits are returned via the eax and edx registers
\item floating point types are returned via the st0 register
\end{itemize}


\paragraph{Stack layout}

Stack directly after function prolog:\\

\begin{figure}[h]
\begin{tabular}{5|3|1 1}
\hhline{~-~~}
                                  & \vdots                     &                                &                              \\
\hhline{~=~~}
local data                        &                            &                                & \mrrbrace{5}{caller's frame} \\
\hhline{~-~~}
\mrlbrace{3}{parameter area}      & \ldots                     & \mrrbrace{3}{stack parameters} &                              \\
                                  & \ldots                     &                                &                              \\
                                  & \ldots                     &                                &                              \\
\hhline{~-~~}
                                  & return address             &                                &                              \\
\hhline{~=~~}
local data                        &                            &                                & \mrrbrace{3}{current frame}  \\
\hhline{~-~~}
parameter area                    &                            &                                &                              \\
\hhline{~-~~}
                                  & \vdots                     &                                &                              \\
\hhline{~-~~}
\end{tabular}
\caption{Stack layout on x86 pascal calling convention}
\end{figure}


\newpage

\subsubsection{plan9call}

\paragraph{Registers and register usage}

\begin{table}[h]
\begin{tabular}{3 B}
\hline
Name          & Brief description\\
\hline
{\bf eax}     & scratch, return value\\
{\bf ebx}     & scratch\\
{\bf ecx}     & scratch\\
{\bf edx}     & scratch\\
{\bf esi}     & scratch\\
{\bf edi}     & scratch\\
{\bf ebp}     & scratch\\
{\bf esp}     & stack pointer\\
{\bf st0}     & scratch, floating point return value\\
{\bf st1-st7} & scratch\\
\hline
\end{tabular}
\caption{Register usage on x86 plan9call calling convention}
\end{table}

\paragraph{Parameter passing}

\begin{itemize}
\item stack parameter order: right-to-left
\item caller cleans up the stack%@@@ doesn't belong to "parameter passing"
\item all parameters are pushed onto the stack
\end{itemize}

\pagebreak

\paragraph{Return values}

\begin{itemize}
\item return values of pointer or integral type (\textless=\ 32 bits) are returned via the eax register
\item integers \textgreater\ 32 bits or structures are returned by the caller allocating the space and
passing a pointer to the callee as a new, implicit first parameter (this means, on the stack)
\item floating point types are returned via the st0 register (called F0 in plan9 8a's terms)
\end{itemize}


\paragraph{Stack layout}

Stack directly after function prolog:\\

\begin{figure}[h]
\begin{tabular}{5|3|1 1}
\hhline{~-~~}
                                  & \vdots                     &                                &                              \\
\hhline{~=~~}
local data                        &                            &                                & \mrrbrace{5}{caller's frame} \\
\hhline{~-~~}
\mrlbrace{3}{parameter area}      & \ldots                     & \mrrbrace{3}{stack parameters} &                              \\
                                  & \ldots                     &                                &                              \\
                                  & \ldots                     &                                &                              \\
\hhline{~-~~}
                                  & return address             &                                &                              \\
\hhline{~=~~}
local data                        &                            &                                & \mrrbrace{3}{current frame}  \\
\hhline{~-~~}
parameter area                    &                            &                                &                              \\
\hhline{~-~~}
                                  & \vdots                     &                                &                              \\
\hhline{~-~~}
\end{tabular}
\\
\\
\\
\caption{Stack layout on x86 plan9call calling convention}
\end{figure}
