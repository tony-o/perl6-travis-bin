%//////////////////////////////////////////////////////////////////////////////
%
% Copyright (c) 2007,2009 Daniel Adler <dadler@uni-goettingen.de>, 
%                         Tassilo Philipp <tphilipp@potion-studios.com>
%
% Permission to use, copy, modify, and distribute this software for any
% purpose with or without fee is hereby granted, provided that the above
% copyright notice and this permission notice appear in all copies.
%
% THE SOFTWARE IS PROVIDED "AS IS" AND THE AUTHOR DISCLAIMS ALL WARRANTIES
% WITH REGARD TO THIS SOFTWARE INCLUDING ALL IMPLIED WARRANTIES OF
% MERCHANTABILITY AND FITNESS. IN NO EVENT SHALL THE AUTHOR BE LIABLE FOR
% ANY SPECIAL, DIRECT, INDIRECT, OR CONSEQUENTIAL DAMAGES OR ANY DAMAGES
% WHATSOEVER RESULTING FROM LOSS OF USE, DATA OR PROFITS, WHETHER IN AN
% ACTION OF CONTRACT, NEGLIGENCE OR OTHER TORTIOUS ACTION, ARISING OUT OF
% OR IN CONNECTION WITH THE USE OR PERFORMANCE OF THIS SOFTWARE.
%
%//////////////////////////////////////////////////////////////////////////////

\newpage

% ==================================================
% Calling Conventions
% ==================================================

\section{Calling Conventions}

\paragraph{Before we go any further\ldots}

It is important to understand that this section isn't a general
purpose description of the present calling conventions.
It merely explains the calling conventions {\bf for the parameter/return types
supported by \dc}, not for aggregates (structures, unions and classes), SIMD
data types (\_\_m64, \_\_m128, \_\_m128i, \_\_m128d), etc.\\
We strongly advise the reader not to use this document as a general purpose
calling convention reference.

%//////////////////////////////////////////////////////////////////////////////
%
% Copyright (c) 2007,2009 Daniel Adler <dadler@uni-goettingen.de>, 
%                         Tassilo Philipp <tphilipp@potion-studios.com>
%
% Permission to use, copy, modify, and distribute this software for any
% purpose with or without fee is hereby granted, provided that the above
% copyright notice and this permission notice appear in all copies.
%
% THE SOFTWARE IS PROVIDED "AS IS" AND THE AUTHOR DISCLAIMS ALL WARRANTIES
% WITH REGARD TO THIS SOFTWARE INCLUDING ALL IMPLIED WARRANTIES OF
% MERCHANTABILITY AND FITNESS. IN NO EVENT SHALL THE AUTHOR BE LIABLE FOR
% ANY SPECIAL, DIRECT, INDIRECT, OR CONSEQUENTIAL DAMAGES OR ANY DAMAGES
% WHATSOEVER RESULTING FROM LOSS OF USE, DATA OR PROFITS, WHETHER IN AN
% ACTION OF CONTRACT, NEGLIGENCE OR OTHER TORTIOUS ACTION, ARISING OUT OF
% OR IN CONNECTION WITH THE USE OR PERFORMANCE OF THIS SOFTWARE.
%
%//////////////////////////////////////////////////////////////////////////////

% ==================================================
% x86 
% ==================================================
\subsection{x86 Calling Conventions}


\paragraph{Overview}

There are numerous different calling conventions on the x86 processor
architecture, like cdecl \cite{x86cdecl}, MS fastcall \cite{x86Winfastcall}, GNU
fastcall \cite{x86GNUfastcall}, Borland fastcall \cite{x86Borlandfastcall}, Watcom
fastcall \cite{x86Watcomfastcall}, Win32 stdcall \cite{x86Winstdcall}, MS thiscall
\cite{x86Winthiscall}, GNU thiscall \cite{x86GNUthiscall}, the pascal calling
convention \cite{x86Pascal} and a cdecl-like version for Plan9 \cite{x86Plan9}
(dubbed plan9call by us), etc.


\paragraph{\product{dyncall} support}

Currently cdecl, stdcall, fastcall (MS and GNU), thiscall (MS and GNU) and
plan9call are supported.\\
\\


\subsubsection{cdecl}

\paragraph{Registers and register usage}

\begin{table}[h]
\begin{tabular}{3 B}
\hline
Name          & Brief description\\
\hline
{\bf eax}     & scratch, return value\\
{\bf ebx}     & permanent\\
{\bf ecx}     & scratch\\
{\bf edx}     & scratch, return value\\
{\bf esi}     & permanent\\
{\bf edi}     & permanent\\
{\bf ebp}     & permanent\\
{\bf esp}     & stack pointer\\
{\bf st0}     & scratch, floating point return value\\
{\bf st1-st7} & scratch\\
\hline
\end{tabular}
\caption{Register usage on x86 cdecl calling convention}
\end{table}


\pagebreak

\paragraph{Parameter passing}

\begin{itemize}
\item stack parameter order: right-to-left
\item caller cleans up the stack
\item all arguments are pushed onto the stack
\end{itemize}

\paragraph{Return values}

\begin{itemize}
\item return values of pointer or integral type (\textless=\ 32 bits) are returned via the eax register
\item integers \textgreater\ 32 bits are returned via the eax and edx registers
\item floating point types are returned via the st0 register
\end{itemize}


\paragraph{Stack layout}

Stack directly after function prolog:\\

\begin{figure}[h]
\begin{tabular}{5|3|1 1}
\hhline{~-~~}
                                  & \vdots                     &                                &                              \\
\hhline{~=~~}
local data                        &                            &                                & \mrrbrace{5}{caller's frame} \\
\hhline{~-~~}
\mrlbrace{3}{parameter area}      & \ldots                     & \mrrbrace{3}{stack parameters} &                              \\
                                  & \ldots                     &                                &                              \\
                                  & \ldots                     &                                &                              \\
\hhline{~-~~}
                                  & return address             &                                &                              \\
\hhline{~=~~}
local data                        &                            &                                & \mrrbrace{3}{current frame}  \\
\hhline{~-~~}
parameter area                    &                            &                                &                              \\
\hhline{~-~~}
                                  & \vdots                     &                                &                              \\
\hhline{~-~~}
\end{tabular}
\caption{Stack layout on x86 cdecl calling convention}
\end{figure}


\pagebreak

\subsubsection{MS fastcall}

\paragraph{Registers and register usage}

\begin{table}[h]
\begin{tabular}{3 B}
\hline
Name          & Brief description\\
\hline
{\bf eax}     & scratch, return value\\
{\bf ebx}     & permanent\\
{\bf ecx}     & scratch, parameter 0\\
{\bf edx}     & scratch, parameter 1, return value\\
{\bf esi}     & permanent\\
{\bf edi}     & permanent\\
{\bf ebp}     & permanent\\
{\bf esp}     & stack pointer\\
{\bf st0}     & scratch, floating point return value\\
{\bf st1-st7} & scratch\\
\hline
\end{tabular}
\caption{Register usage on x86 fastcall (MS) calling convention}
\end{table}


\pagebreak

\paragraph{Parameter passing}

\begin{itemize}
\item stack parameter order: right-to-left
\item called function cleans up the stack
\item first two integers/pointers (\textless=\ 32bit) are passed via ecx and edx (even if preceded by other arguments)
\item integer types 64 bits in size @@@ ? first in edx:eax ?
\item if first argument is a 64 bit integer, it is passed via ecx and edx
\item all other parameters are pushed onto the stack
\end{itemize}

\paragraph{Return values}

\begin{itemize}
\item return values of pointer or integral type (\textless=\ 32 bits) are returned via the eax register
\item integers \textgreater\ 32 bits are returned via the eax and edx registers@@@verify
\item floating point types are returned via the st0 register@@@ really ?
\end{itemize}


\paragraph{Stack layout}

Stack directly after function prolog:\\

\begin{figure}[h]
\begin{tabular}{5|3|1 1}
\hhline{~-~~}
                                  & \vdots                     &                                &                              \\
\hhline{~=~~}
local data                        &                            &                                & \mrrbrace{5}{caller's frame} \\
\hhline{~-~~}
\mrlbrace{3}{parameter area}      & \ldots                     & \mrrbrace{3}{stack parameters} &                              \\
                                  & \ldots                     &                                &                              \\
                                  & \ldots                     &                                &                              \\
\hhline{~-~~}
                                  & return address             &                                &                              \\
\hhline{~=~~}
local data                        &                            &                                & \mrrbrace{3}{current frame}  \\
\hhline{~-~~}
parameter area                    &                            &                                &                              \\
\hhline{~-~~}
                                  & \vdots                     &                                &                              \\
\hhline{~-~~}
\end{tabular}
\caption{Stack layout on x86 fastcall (MS) calling convention}
\end{figure}


\pagebreak

\subsubsection{GNU fastcall}

\paragraph{Registers and register usage}

\begin{table}[h]
\begin{tabular}{3 B}
\hline
Name          & Brief description\\
\hline
{\bf eax}     & scratch, return value\\
{\bf ebx}     & permanent\\
{\bf ecx}     & scratch, parameter 0\\
{\bf edx}     & scratch, parameter 1, return value\\
{\bf esi}     & permanent\\
{\bf edi}     & permanent\\
{\bf ebp}     & permanent\\
{\bf esp}     & stack pointer\\
{\bf st0}     & scratch, floating point return value\\
{\bf st1-st7} & scratch\\
\hline
\end{tabular}
\caption{Register usage on x86 fastcall (GNU) calling convention}
\end{table}

\paragraph{Parameter passing}

\begin{itemize}
\item stack parameter order: right-to-left
\item called function cleans up the stack
\item first two integers/pointers (\textless=\ 32bit) are passed via ecx and edx (even if preceded by other arguments)
\item if first argument is a 64 bit integer, it is pushed on the stack and the two registers are skipped 
\item all other parameters are pushed onto the stack
\end{itemize}


\paragraph{Return values}

\begin{itemize}
\item return values of pointer or integral type (\textless=\ 32 bits) are returned via the eax register.
\item integers \textgreater\ 32 bits are returned via the eax and edx registers.
\item floating point types are returned via the st0.
\end{itemize}


\pagebreak

\paragraph{Stack layout}

Stack directly after function prolog:\\

\begin{figure}[h]
\begin{tabular}{5|3|1 1}
\hhline{~-~~}
                                  & \vdots                     &                                &                              \\
\hhline{~=~~}
local data                        &                            &                                & \mrrbrace{5}{caller's frame} \\
\hhline{~-~~}
\mrlbrace{3}{parameter area}      & \ldots                     & \mrrbrace{3}{stack parameters} &                              \\
                                  & \ldots                     &                                &                              \\
                                  & \ldots                     &                                &                              \\
\hhline{~-~~}
                                  & return address             &                                &                              \\
\hhline{~=~~}
local data                        &                            &                                & \mrrbrace{3}{current frame}  \\
\hhline{~-~~}
parameter area                    &                            &                                &                              \\
\hhline{~-~~}
                                  & \vdots                     &                                &                              \\
\hhline{~-~~}
\end{tabular}
\caption{Stack layout on x86 fastcall (GNU) calling convention}
\end{figure}


\subsubsection{Borland fastcall}

\paragraph{Registers and register usage}

\begin{table}[h]
\begin{tabular}{3 B}
\hline
Name          & Brief description\\
\hline
{\bf eax}     & scratch, parameter 0, return value\\
{\bf ebx}     & permanent\\
{\bf ecx}     & scratch, parameter 2\\
{\bf edx}     & scratch, parameter 1, return value\\
{\bf esi}     & permanent\\
{\bf edi}     & permanent\\
{\bf ebp}     & permanent\\
{\bf esp}     & stack pointer\\
{\bf st0}     & scratch, floating point return value\\
{\bf st1-st7} & scratch\\
\hline
\end{tabular}
\caption{Register usage on x86 fastcall (Borland) calling convention}
\end{table}

\paragraph{Parameter passing}

\begin{itemize}
\item stack parameter order: left-to-right
\item called function cleans up the stack
\item first three integers/pointers (\textless=\ 32bit) are passed via eax, ecx and edx (even if preceded by other arguments@@@?)
\item integer types 64 bits in size @@@ ?
\item all other parameters are pushed onto the stack
\end{itemize}


\pagebreak

\paragraph{Return values}

\begin{itemize}
\item return values of pointer or integral type (\textless=\ 32 bits) are returned via the eax register
\item integers \textgreater\ 32 bits are returned via the eax and edx registers@@@ verify
\item floating point types are returned via the st0 register@@@ really ?
\end{itemize}



\paragraph{Stack layout}

Stack directly after function prolog:\\

\begin{figure}[h]
\begin{tabular}{5|3|1 1}
\hhline{~-~~}
                                  & \vdots                     &                                &                              \\
\hhline{~=~~}
local data                        &                            &                                & \mrrbrace{5}{caller's frame} \\
\hhline{~-~~}
\mrlbrace{3}{parameter area}      & \ldots                     & \mrrbrace{3}{stack parameters} &                              \\
                                  & \ldots                     &                                &                              \\
                                  & \ldots                     &                                &                              \\
\hhline{~-~~}
                                  & return address             &                                &                              \\
\hhline{~=~~}
local data                        &                            &                                & \mrrbrace{3}{current frame}  \\
\hhline{~-~~}
parameter area                    &                            &                                &                              \\
\hhline{~-~~}
                                  & \vdots                     &                                &                              \\
\hhline{~-~~}
\end{tabular}
\caption{Stack layout on x86 fastcall (Borland) calling convention}
\end{figure}


\subsubsection{Watcom fastcall}


\paragraph{Registers and register usage}

\begin{table}[h]
\begin{tabular}{3 B}
\hline
Name          & Brief description\\
\hline
{\bf eax}     & scratch, parameter 0, return value@@@\\
{\bf ebx}     & scratch when used for parameter, parameter 2\\
{\bf ecx}     & scratch when used for parameter, parameter 3\\
{\bf edx}     & scratch when used for parameter, parameter 1, return value@@@\\
{\bf esi}     & scratch when used for return pointer @@@??\\
{\bf edi}     & permanent\\
{\bf ebp}     & permanent\\
{\bf esp}     & stack pointer\\
{\bf st0}     & scratch, floating point return value\\
{\bf st1-st7} & scratch\\
\hline
\end{tabular}
\caption{Register usage on x86 fastcall (Watcom) calling convention}
\end{table}

\paragraph{Parameter passing}

\begin{itemize}
\item stack parameter order: right-to-left
\item called function cleans up the stack
\item first four integers/pointers (\textless=\ 32bit) are passed via eax, edx, ebx and ecx (even if preceded by other arguments@@@?)
\item integer types 64 bits in size @@@ ?
\item all other parameters are pushed onto the stack
\end{itemize}


\paragraph{Return values}

\begin{itemize}
\item return values of pointer or integral type (\textless=\ 32 bits) are returned via the eax register@@@verify, I thnik its esi?
\item integers \textgreater\ 32 bits are returned via the eax and edx registers@@@ verify
\item floating point types are returned via the st0 register@@@ really ?
\end{itemize}


\paragraph{Stack layout}

Stack directly after function prolog:\\

\begin{figure}[h]
\begin{tabular}{5|3|1 1}
\hhline{~-~~}
                                  & \vdots                     &                                &                              \\
\hhline{~=~~}
local data                        &                            &                                & \mrrbrace{5}{caller's frame} \\
\hhline{~-~~}
\mrlbrace{3}{parameter area}      & \ldots                     & \mrrbrace{3}{stack parameters} &                              \\
                                  & \ldots                     &                                &                              \\
                                  & \ldots                     &                                &                              \\
\hhline{~-~~}
                                  & return address             &                                &                              \\
\hhline{~=~~}
local data                        &                            &                                & \mrrbrace{3}{current frame}  \\
\hhline{~-~~}
parameter area                    &                            &                                &                              \\
\hhline{~-~~}
                                  & \vdots                     &                                &                              \\
\hhline{~-~~}
\end{tabular}
\caption{Stack layout on x86 fastcall (Watcom) calling convention}
\end{figure}



\subsubsection{win32 stdcall}

\paragraph{Registers and register usage}

\begin{table}[h]
\begin{tabular}{3 B}
\hline
Name          & Brief description\\
\hline
{\bf eax}     & scratch, return value\\
{\bf ebx}     & permanent\\
{\bf ecx}     & scratch\\
{\bf edx}     & scratch, return value\\
{\bf esi}     & permanent\\
{\bf edi}     & permanent\\
{\bf ebp}     & permanent\\
{\bf esp}     & stack pointer\\
{\bf st0}     & scratch, floating point return value\\
{\bf st1-st7} & scratch\\
\hline
\end{tabular}
\caption{Register usage on x86 stdcall calling convention}
\end{table}

\paragraph{Parameter passing}

\begin{itemize}
\item Stack parameter order: right-to-left
\item Called function cleans up the stack
\item All parameters are pushed onto the stack
\item Stack is usually 4 byte aligned (GCC \textgreater=\ 3.x seems to use a 16byte alignement@@@)
\item Function name is decorated by prepending an underscore character and appending a '@' character and the number of bytes of stack space required
\end{itemize}


\paragraph{Return values}

\begin{itemize}
\item return values of pointer or integral type (\textless=\ 32 bits) are returned via the eax register
\item integers \textgreater\ 32 bits are returned via the eax and edx registers
\item floating point types are returned via the st0 register
\end{itemize}


\paragraph{Stack layout}

Stack directly after function prolog:\\

\begin{figure}[h]
\begin{tabular}{5|3|1 1}
\hhline{~-~~}
                                  & \vdots                     &                                &                              \\
\hhline{~=~~}
local data                        &                            &                                & \mrrbrace{5}{caller's frame} \\
\hhline{~-~~}
\mrlbrace{3}{parameter area}      & \ldots                     & \mrrbrace{3}{stack parameters} &                              \\
                                  & \ldots                     &                                &                              \\
                                  & \ldots                     &                                &                              \\
\hhline{~-~~}
                                  & return address             &                                &                              \\
\hhline{~=~~}
local data                        &                            &                                & \mrrbrace{3}{current frame}  \\
\hhline{~-~~}
parameter area                    &                            &                                &                              \\
\hhline{~-~~}
                                  & \vdots                     &                                &                              \\
\hhline{~-~~}
\end{tabular}
\caption{Stack layout on x86 stdcall calling convention}
\end{figure}

\subsubsection{MS thiscall}

\paragraph{Registers and register usage}

\begin{table}[h]
\begin{tabular}{3 B}
\hline
Name          & Brief description\\
\hline
{\bf eax}     & scratch, return value\\
{\bf ebx}     & permanent\\
{\bf ecx}     & scratch, parameter 0\\
{\bf edx}     & scratch, return value\\
{\bf esi}     & permanent\\
{\bf edi}     & permanent\\
{\bf ebp}     & permanent\\
{\bf esp}     & stack pointer\\
{\bf st0}     & scratch, floating point return value\\
{\bf st1-st7} & scratch\\
\hline
\end{tabular}
\caption{Register usage on x86 thiscall (MS) calling convention}
\end{table}

\newpage


\paragraph{Parameter passing}

\begin{itemize}
\item stack parameter order: right-to-left
\item called function cleans up the stack
\item first parameter (this pointer) is passed via ecx
\item all other parameters are pushed onto the stack
\item Function name is decorated by prepending a '@' character and appending a '@' character and the number of bytes (decimal) of stack space required
\end{itemize}


\paragraph{Return values}

\begin{itemize}
\item return values of pointer or integral type (\textless=\ 32 bits) are returned via the eax register
\item integers \textgreater\ 32 bits are returned via the eax and edx registers@@@verify
\item floating point types are returned via the st0 register@@@ really ?
\end{itemize}


\paragraph{Stack layout}

Stack directly after function prolog:\\

\begin{figure}[h]
\begin{tabular}{5|3|1 1}
\hhline{~-~~}
                                  & \vdots                     &                                &                              \\
\hhline{~=~~}
local data                        &                            &                                & \mrrbrace{5}{caller's frame} \\
\hhline{~-~~}
\mrlbrace{3}{parameter area}      & \ldots                     & \mrrbrace{3}{stack parameters} &                              \\
                                  & \ldots                     &                                &                              \\
                                  & \ldots                     &                                &                              \\
\hhline{~-~~}
                                  & return address             &                                &                              \\
\hhline{~=~~}
local data                        &                            &                                & \mrrbrace{3}{current frame}  \\
\hhline{~-~~}
parameter area                    &                            &                                &                              \\
\hhline{~-~~}
                                  & \vdots                     &                                &                              \\
\hhline{~-~~}
\end{tabular}
\caption{Stack layout on x86 thiscall (MS) calling convention}
\end{figure}



\subsubsection{GNU thiscall}

\paragraph{Registers and register usage}

\begin{table}[h]
\begin{tabular}{3 B}
\hline
Name          & Brief description\\
\hline
{\bf eax}     & scratch, return value\\
{\bf ebx}     & permanent\\
{\bf ecx}     & scratch\\
{\bf edx}     & scratch, return value\\
{\bf esi}     & permanent\\
{\bf edi}     & permanent\\
{\bf ebp}     & permanent\\
{\bf esp}     & stack pointer\\
{\bf st0}     & scratch, floating point return value\\
{\bf st1-st7} & scratch\\
\hline
\end{tabular}
\caption{Register usage on x86 thiscall (GNU) calling convention}
\end{table}

\paragraph{Parameter passing}

\begin{itemize}
\item stack parameter order: right-to-left
\item caller cleans up the stack
\item all parameters are pushed onto the stack
\end{itemize}


\paragraph{Return values}

\begin{itemize}
\item return values of pointer or integral type (\textless=\ 32 bits) are returned via the eax register
\item integers \textgreater\ 32 bits are returned via the eax and edx registers@@@verify
\item floating point types are returned via the st0 register@@@ really ?
\end{itemize}


\paragraph{Stack layout}

Stack directly after function prolog:\\

\begin{figure}[h]
\begin{tabular}{5|3|1 1}
\hhline{~-~~}
                                  & \vdots                     &                                &                              \\
\hhline{~=~~}
local data                        &                            &                                & \mrrbrace{5}{caller's frame} \\
\hhline{~-~~}
\mrlbrace{3}{parameter area}      & \ldots                     & \mrrbrace{3}{stack parameters} &                              \\
                                  & \ldots                     &                                &                              \\
                                  & \ldots                     &                                &                              \\
\hhline{~-~~}
                                  & return address             &                                &                              \\
\hhline{~=~~}
local data                        &                            &                                & \mrrbrace{3}{current frame}  \\
\hhline{~-~~}
parameter area                    &                            &                                &                              \\
\hhline{~-~~}
                                  & \vdots                     &                                &                              \\
\hhline{~-~~}
\end{tabular}
\caption{Stack layout on x86 thiscall (GNU) calling convention}
\end{figure}



\subsubsection{pascal}

The best known uses of the pascal calling convention are the 16 bit OS/2 APIs, Microsoft Windows 3.x and Borland Delphi 1.x.

\paragraph{Registers and register usage}

\begin{table}[h]
\begin{tabular}{3 B}
\hline
Name          & Brief description\\
\hline
{\bf eax}     & scratch, return value\\
{\bf ebx}     & permanent\\
{\bf ecx}     & scratch\\
{\bf edx}     & scratch, return value\\
{\bf esi}     & permanent\\
{\bf edi}     & permanent\\
{\bf ebp}     & permanent\\
{\bf esp}     & stack pointer\\
{\bf st0}     & scratch, floating point return value\\
{\bf st1-st7} & scratch\\
\hline
\end{tabular}
\caption{Register usage on x86 pascal calling convention}
\end{table}

\paragraph{Parameter passing}

\begin{itemize}
\item stack parameter order: left-to-right
\item called function cleans up the stack
\item all parameters are pushed onto the stack
\end{itemize}


\paragraph{Return values}

\begin{itemize}
\item return values of pointer or integral type (\textless=\ 32 bits) are returned via the eax register
\item integers \textgreater\ 32 bits are returned via the eax and edx registers
\item floating point types are returned via the st0 register
\end{itemize}


\paragraph{Stack layout}

Stack directly after function prolog:\\

\begin{figure}[h]
\begin{tabular}{5|3|1 1}
\hhline{~-~~}
                                  & \vdots                     &                                &                              \\
\hhline{~=~~}
local data                        &                            &                                & \mrrbrace{5}{caller's frame} \\
\hhline{~-~~}
\mrlbrace{3}{parameter area}      & \ldots                     & \mrrbrace{3}{stack parameters} &                              \\
                                  & \ldots                     &                                &                              \\
                                  & \ldots                     &                                &                              \\
\hhline{~-~~}
                                  & return address             &                                &                              \\
\hhline{~=~~}
local data                        &                            &                                & \mrrbrace{3}{current frame}  \\
\hhline{~-~~}
parameter area                    &                            &                                &                              \\
\hhline{~-~~}
                                  & \vdots                     &                                &                              \\
\hhline{~-~~}
\end{tabular}
\caption{Stack layout on x86 pascal calling convention}
\end{figure}


\newpage

\subsubsection{plan9call}

\paragraph{Registers and register usage}

\begin{table}[h]
\begin{tabular}{3 B}
\hline
Name          & Brief description\\
\hline
{\bf eax}     & scratch, return value\\
{\bf ebx}     & scratch\\
{\bf ecx}     & scratch\\
{\bf edx}     & scratch\\
{\bf esi}     & scratch\\
{\bf edi}     & scratch\\
{\bf ebp}     & scratch\\
{\bf esp}     & stack pointer\\
{\bf st0}     & scratch, floating point return value\\
{\bf st1-st7} & scratch\\
\hline
\end{tabular}
\caption{Register usage on x86 plan9call calling convention}
\end{table}

\paragraph{Parameter passing}

\begin{itemize}
\item stack parameter order: right-to-left
\item caller cleans up the stack%@@@ doesn't belong to "parameter passing"
\item all parameters are pushed onto the stack
\end{itemize}

\pagebreak

\paragraph{Return values}

\begin{itemize}
\item return values of pointer or integral type (\textless=\ 32 bits) are returned via the eax register
\item integers \textgreater\ 32 bits or structures are returned by the caller allocating the space and
passing a pointer to the callee as a new, implicit first parameter (this means, on the stack)
\item floating point types are returned via the st0 register (called F0 in plan9 8a's terms)
\end{itemize}


\paragraph{Stack layout}

Stack directly after function prolog:\\

\begin{figure}[h]
\begin{tabular}{5|3|1 1}
\hhline{~-~~}
                                  & \vdots                     &                                &                              \\
\hhline{~=~~}
local data                        &                            &                                & \mrrbrace{5}{caller's frame} \\
\hhline{~-~~}
\mrlbrace{3}{parameter area}      & \ldots                     & \mrrbrace{3}{stack parameters} &                              \\
                                  & \ldots                     &                                &                              \\
                                  & \ldots                     &                                &                              \\
\hhline{~-~~}
                                  & return address             &                                &                              \\
\hhline{~=~~}
local data                        &                            &                                & \mrrbrace{3}{current frame}  \\
\hhline{~-~~}
parameter area                    &                            &                                &                              \\
\hhline{~-~~}
                                  & \vdots                     &                                &                              \\
\hhline{~-~~}
\end{tabular}
\\
\\
\\
\caption{Stack layout on x86 plan9call calling convention}
\end{figure}
\newpage
%//////////////////////////////////////////////////////////////////////////////
%
% Copyright (c) 2007,2009 Daniel Adler <dadler@uni-goettingen.de>, 
%                         Tassilo Philipp <tphilipp@potion-studios.com>
%
% Permission to use, copy, modify, and distribute this software for any
% purpose with or without fee is hereby granted, provided that the above
% copyright notice and this permission notice appear in all copies.
%
% THE SOFTWARE IS PROVIDED "AS IS" AND THE AUTHOR DISCLAIMS ALL WARRANTIES
% WITH REGARD TO THIS SOFTWARE INCLUDING ALL IMPLIED WARRANTIES OF
% MERCHANTABILITY AND FITNESS. IN NO EVENT SHALL THE AUTHOR BE LIABLE FOR
% ANY SPECIAL, DIRECT, INDIRECT, OR CONSEQUENTIAL DAMAGES OR ANY DAMAGES
% WHATSOEVER RESULTING FROM LOSS OF USE, DATA OR PROFITS, WHETHER IN AN
% ACTION OF CONTRACT, NEGLIGENCE OR OTHER TORTIOUS ACTION, ARISING OUT OF
% OR IN CONNECTION WITH THE USE OR PERFORMANCE OF THIS SOFTWARE.
%
%//////////////////////////////////////////////////////////////////////////////

% ==================================================
% x64
% ==================================================
\subsection{x64 Calling Convention}


\paragraph{Overview}

The x64 (64bit) architecture designed by AMD is based on Intel's x86 (32bit)
architecture, supporting it natively. It is sometimes referred to as x86-64,
AMD64, or, cloned by Intel, EM64T or Intel64.\\
On this processor, a word is defined to be 16 bits in size, a dword 32 bits
and a qword 64 bits. Note that this is due to historical reasons (terminology
didn't change with the introduction of 32 and 64 bit processors).\\
The x64 calling convention for MS Windows \cite{x64Win} differs from the
SystemV x64 calling convention \cite{x64SysV} used by Linux/*BSD/...
Note that this is not the only difference between these operating systems. The
64 bit programming model in use by 64 bit windows is LLP64, meaning that the C
types int and long remain 32 bits in size, whereas long long becomes 64 bits.
Under Linux/*BSD/... it's LP64.\\
\\
Compared to the x86 architecture, the 64 bit versions of the registers are
called rax, rbx, etc.. Furthermore, there are eight new general purpose
registers r8-r15.



\paragraph{\product{dyncall} support}

\product{dyncall} supports the MS Windows and System V calling convention.\\
\\



\subsubsection{MS Windows}

\paragraph{Registers and register usage}

\begin{table}[h]
\begin{tabular}{3 B}
\hline
Name                & Brief description\\
\hline
{\bf rax}           & scratch, return value\\
{\bf rbx}           & permanent\\
{\bf rcx}           & scratch, parameter 0 if integer or pointer\\
{\bf rdx}           & scratch, parameter 1 if integer or pointer\\
{\bf rdi}           & permanent\\
{\bf rsi}           & permanent\\
{\bf rbp}           & permanent, may be used ase frame pointer\\
{\bf rsp}           & stack pointer\\
{\bf r8-r9}         & scratch, parameter 2 and 3 if integer or pointer\\
{\bf r10-r11}       & scratch, permanent if required by caller (used for syscall/sysret)\\
{\bf r12-r15}       & permanent\\
{\bf xmm0}          & scratch, floating point parameter 0, floating point return value\\
{\bf xmm1-xmm3}     & scratch, floating point parameters 1-3\\
{\bf xmm4-xmm5}     & scratch, permanent if required by caller\\
{\bf xmm6-xmm15}    & permanent\\
\hline
\end{tabular}
\caption{Register usage on x64 MS Windows platform}
\end{table}

\paragraph{Parameter passing}

\begin{itemize}
\item stack parameter order: right-to-left
\item caller cleans up the stack
\item first 4 integer/pointer parameters are passed via rcx, rdx, r8, r9 (from left to right), others are pushed on stack (there is a
preserve area for the first 4)
\item float and double parameters are passed via xmm0l-xmm3l
\item first 4 parameters are passed via the correct register depending on the parameter type - with mixed float and int parameters,
some registers are left out (e.g. first parameter ends up in rcx or xmm0, second in rdx or xmm1, etc.)
\item parameters in registers are right justified
\item parameters \textless\ 64bits are not zero extended - zero the upper bits contiaining garbage if needed (but they are always
passed as a qword)
\item parameters \textgreater\ 64 bit are passed by reference
\item if callee takes address of a parameter, first 4 parameters must be dumped (to the reserved space on the stack) - for
floating point parameters, value must be stored in integer AND floating point register
\item caller cleans up the stack, not the callee (like cdecl)
\item stack is always 16byte aligned - since return address is 64 bits in size, stacks with an odd number of parameters are
already aligned
\item ellipsis calls take floating point values in int and float registers (single precision floats are promoted to double precision
as defined for ellipsis calls)
\item if size of parameters \textgreater\ 1 page of memory (usually between 4k and 64k), chkstk must be called
\end{itemize}


\paragraph{Return values}

\begin{itemize}
\item return values of pointer or integral type (\textless=\ 64 bits) are returned via the rax register
\item floating point types are returned via the xmm0 register
\item for types \textgreater\ 64 bits, a secret first parameter with an address to the return value is passed
\end{itemize}


\paragraph{Stack layout}

Stack frame is always 16-byte aligned. Stack directly after function prolog:\\

\begin{figure}[h]
\begin{tabular}{5|3|1 1}
\hhline{~-~~}
                                  & \vdots                     &                                &                              \\
\hhline{~=~~}
local data                        &                            &                                & \mrrbrace{9}{caller's frame} \\
\hhline{~-~~}
\mrlbrace{7}{parameter area}      & \ldots                     & \mrrbrace{3}{stack parameters} &                              \\
                                  & \ldots                     &                                &                              \\
                                  & \ldots                     &                                &                              \\
                                  & r9 or xmm3                 & \mrrbrace{4}{spill area}       &                              \\
                                  & r8 or xmm2                 &                                &                              \\
                                  & rdx or xmm1                &                                &                              \\
                                  & rcx or xmm0                &                                &                              \\
\hhline{~-~~}
                                  & return address             &                                &                              \\
\hhline{~=~~}
local data                        &                            &                                & \mrrbrace{3}{current frame}  \\
\hhline{~-~~}
parameter area                    &                            &                                &                              \\
\hhline{~-~~}
                                  & \vdots                     &                                &                              \\
\hhline{~-~~}
\end{tabular}
\caption{Stack layout on x64 Microsoft platform}
\end{figure}



\newpage

\subsubsection{System V (Linux / *BSD / MacOS X)}

\paragraph{Registers and register usage}

\begin{table}[h]
\begin{tabular}{3 B}
\hline
Name                & Brief description\\
\hline
{\bf rax}           & scratch, return value\\
{\bf rbx}           & permanent\\
{\bf rcx}           & scratch, parameter 3 if integer or pointer\\
{\bf rdx}           & scratch, parameter 2 if integer or pointer, return value\\
{\bf rdi}           & scratch, parameter 0 if integer or pointer\\
{\bf rsi}           & scratch, parameter 1 if integer or pointer\\
{\bf rbp}           & permanent, may be used ase frame pointer\\
{\bf rsp}           & stack pointer\\
{\bf r8-r9}         & scratch, parameter 4 and 5 if integer or pointer\\
{\bf r10-r11}       & scratch\\
{\bf r12-r15}       & permanent\\
{\bf xmm0}          & scratch, floating point parameters 0, floating point return value\\
{\bf xmm1-xmm7}     & scratch, floating point parameters 1-7\\
{\bf xmm8-xmm15}    & scratch\\
{\bf st0-st1}       & scratch, 16 byte floating point return value\\
{\bf st2-st7}       & scratch\\
\hline
\end{tabular}
\caption{Register usage on x64 System V (Linux/*BSD)}
\end{table}

\paragraph{Parameter passing}

\begin{itemize}
\item stack parameter order: right-to-left
\item caller cleans up the stack
\item first 6 integer/pointer parameters are passed via rdi, rsi, rdx, rcx, r8, r9
\item first 8 floating point parameters \textless=\ 64 bits are passed via xmm0l-xmm7l
\item parameters in registers are right justified
\item parameters that are not passed via registers are pushed onto the stack
\item parameters \textless\ 64bits are not zero extended - zero the upper bits contiaining garbage if needed (but they are always
passed as a qword)
\item integer/pointer parameters \textgreater\ 64 bit are passed via 2 registers
\item if callee takes address of a parameter, number of used xmm registers is passed silently in al (passed number mustn't be
exact but an upper bound on the number of used xmm registers)
\item stack is always 16byte aligned - since return address is 64 bits in size, stacks with an odd number of parameters are
already aligned
\end{itemize}


\paragraph{Return values}

\begin{itemize}
\item return values of pointer or integral type (\textless=\ 64 bits) are returned via the rax register
\item floating point types are returned via the xmm0 register
\item for types \textgreater\ 64 bits, a secret first parameter with an address to the return value is passed - the passed in address
will be returned in rax
\item floating point values \textgreater\ 64 bits are returned via st0 and st1
\end{itemize}


\paragraph{Stack layout}

Stack frame is always 16-byte aligned. Note that there is no spill area.
Stack directly after function prolog:\\

\begin{figure}[h]
\begin{tabular}{5|3|1 1}
\hhline{~-~~}
                                  & \vdots                     &                                &                              \\
\hhline{~=~~}
local data                        &                            &                                & \mrrbrace{5}{caller's frame} \\
\hhline{~-~~}
\mrlbrace{3}{parameter area}      & \ldots                     & \mrrbrace{3}{stack parameters} &                              \\
                                  & \ldots                     &                                &                              \\
                                  & \ldots                     &                                &                              \\
\hhline{~-~~}
                                  & return address             &                                &                              \\
\hhline{~=~~}
local data                        &                            &                                & \mrrbrace{3}{current frame}  \\
\hhline{~-~~}
parameter area                    &                            &                                &                              \\
\hhline{~-~~}
                                  & \vdots                     &                                &                              \\
\hhline{~-~~}
\end{tabular}
\caption{Stack layout on x64 System V (Linux/*BSD)}
\end{figure}

\newpage
%//////////////////////////////////////////////////////////////////////////////
%
% Copyright (c) 2007,2009 Daniel Adler <dadler@uni-goettingen.de>, 
%                         Tassilo Philipp <tphilipp@potion-studios.com>
%
% Permission to use, copy, modify, and distribute this software for any
% purpose with or without fee is hereby granted, provided that the above
% copyright notice and this permission notice appear in all copies.
%
% THE SOFTWARE IS PROVIDED "AS IS" AND THE AUTHOR DISCLAIMS ALL WARRANTIES
% WITH REGARD TO THIS SOFTWARE INCLUDING ALL IMPLIED WARRANTIES OF
% MERCHANTABILITY AND FITNESS. IN NO EVENT SHALL THE AUTHOR BE LIABLE FOR
% ANY SPECIAL, DIRECT, INDIRECT, OR CONSEQUENTIAL DAMAGES OR ANY DAMAGES
% WHATSOEVER RESULTING FROM LOSS OF USE, DATA OR PROFITS, WHETHER IN AN
% ACTION OF CONTRACT, NEGLIGENCE OR OTHER TORTIOUS ACTION, ARISING OUT OF
% OR IN CONNECTION WITH THE USE OR PERFORMANCE OF THIS SOFTWARE.
%
%//////////////////////////////////////////////////////////////////////////////

% ==================================================
% PowerPC 32
% ==================================================
\subsection{PowerPC (32bit) Calling Convention}

\paragraph{Overview}

\begin{itemize}
\item Word size is 32 bits
\item Big endian (MSB) and litte endian (LSB) operating modes.
\item Processor operates on floats in double precision floating point arithmetc (IEEE-754) values directly (single precision is converted on the fly)
\item Apple Mac OS X/Darwin PPC is specified in "Mac OS X ABI Function Call Guide"\cite{ppcMacOSX}. It uses Big Endian (MSB).
\item Linux PPC 32-bit ABI is specified in "LSB for PPC"\cite{ppc32LSB} which is based on "System V ABI". It uses Big Endian (MSB).
\item PowerPC EABI is defined in the "PowerPC Embedded Application Binary Interface 32-Bit Implementation".
\end{itemize}


\paragraph{\product{dyncall} support}

\product{Dyncall} and \product{dyncallback} are supported for PowerPC (32bit) Big Endian (MSB) on Darwin (tested on Apple Mac OS X) and Linux, however, fail for *BSD.


\subsubsection{Mac OS X/Darwin}

\paragraph{Registers and register usage}

\begin{table}[h]
\begin{tabular}{3 B}
\hline
Name                & Brief description\\
\hline
{\bf gpr0}          & scratch\\
{\bf gpr1}          & stack pointer\\
{\bf gpr2}          & scratch\\
{\bf gpr3}          & return value, parameter 0 if integer or pointer\\
{\bf gpr4-gpr10}    & return value, parameter 1-7 for integer or pointer parameters\\
{\bf gpr11}         & permanent\\
{\bf gpr12}         & branch target for dynamic code generation\\
{\bf gpr13-31}      & permanent\\
{\bf fpr0}          & scratch\\
{\bf fpr1-fpr13}    & parameter 0-12 for floating point (always double precision)\\
{\bf fpr14-fpr31}   & permanent\\
{\bf v0-v1}         & scratch\\
{\bf v2-v13}        & vector parameters\\
{\bf v14-v19}       & scratch\\
{\bf v20-v31}       & permanent\\
{\bf lr}            & scratch, link-register\\
{\bf ctr}           & scratch, count-register\\
{\bf cr0-cr1}       & scratch\\
{\bf cr2-cr4}       & permanent\\
{\bf cr5-cr7}       & scratch\\
\hline
\end{tabular}
\caption{Register usage on Darwin PowerPC 32-Bit}
\end{table}

\paragraph{Parameter passing}

\begin{itemize}
\item stack parameter order: right-to-left@@@?
\item caller cleans up the stack@@@?
\item the first 8 integer parameters are passed in registers gpr3-gpr10
\item the first 12 floating point parameters are passed in registers fpr1-fpr13
\item if a float parameter is passed via a register, gpr registers are skipped for subsequent integer parameters (based on the size of
the float - 1 register for single precision and 2 for double precision floating point values)
\item the caller pushes subsequent parameters onto the stack
\item for every parameter passed via a register, space is reserved in the stack parameter area (in order to spill the parameters if
needed - e.g. varargs)
\item ellipsis calls take floating point values in int and float registers (single precision floats are promoted to double precision
as defined for ellipsis calls)
\item all nonvector parameters are aligned on 4-byte boundaries
\item vector parameters are aligned on 16-byte boundaries
\item integer parameters \textless\ 32 bit occupy high-order bytes of their 4-byte area
\item composite parameters with size of 1 or 2 bytes occupy low-order bytes of their 4-byte area. INCONSISTENT with other 32-bit PPC
binary interfaces. In AIX and OS 9, padding bytes always follow the data structure
\item composite parameters 3 bytes or larger in size occupy high-order bytes
\end{itemize}


\paragraph{Return values}

\begin{itemize}
\item return values of integer \textless=\ 32bit or pointer type use gpr3
\item 64 bit integers use gpr3 and gpr4 (hiword in gpr3, loword in gpr4)
\item floating point values are returned via fpr1
\item structures \textless=\ 64 bits use gpr3 and gpr4
\item for types \textgreater\ 64 bits, a secret first parameter with an address to the return value is passed
\end{itemize}

\pagebreak

\paragraph{Stack layout}

Stack frame is always 16-byte aligned. Stack directly after function prolog:\\

\begin{figure}[h]
\begin{tabular}{5|3|1 1}
\hhline{~-~~}
                                  & \vdots              &                                      &                               \\
\hhline{~=~~}
local data                        &                     &                                      & \mrrbrace{13}{caller's frame} \\
\hhline{~-~~}
\mrlbrace{6}{parameter area}      & \ldots              & \mrrbrace{3}{stack parameters}       &                               \\
                                  & \ldots              &                                      &                               \\
                                  & \ldots              &                                      &                               \\
                                  & \ldots              & \mrrbrace{3}{spill area (as needed)} &                               \\
                                  & \ldots              &                                      &                               \\
                                  & gpr3 or fpr1        &                                      &                               \\
\hhline{~-~~}
\mrlbrace{6}{linkage area}        & reserved            &                                      &                               \\
                                  & reserved            &                                      &                               \\
                                  & reserved            &                                      &                               \\
                                  & return address      &                                      &                               \\
                                  & reserved for callee &                                      &                               \\
                                  & saved by callee     &                                      &                               \\
\hhline{~=~~}
local data                        &                     &                                      & \mrrbrace{3}{current frame}   \\
\hhline{~-~~}
parameter area                    &                     &                                      &                               \\
\hhline{~-~~}
linkage area                      & \vdots              &                                      &                               \\
\hhline{~-~~}
\end{tabular}
\caption{Stack layout on ppc32 Darwin}
\end{figure}

\subsubsection{System V PPC 32-bit}

\paragraph{Status}

\begin{itemize}
\item C++ this calls do not work.
\item Callbacks don't work on *BSD.
\end{itemize}

\paragraph{Registers and register usage}

\begin{table}[h]
\begin{tabular}{3 B}
\hline
Name                & Brief description\\
\hline
{\bf r0}          & scratch\\
{\bf r1}          & stack pointer\\
{\bf r2}          & system-reserved\\
{\bf r3-r4}       & parameter passing and return value\\
{\bf r5-r10}      & parameter passing\\
{\bf r11-r12}     & scratch\\
{\bf r13}         & Small data area pointer register\\
{\bf r14-r30}     & Local variables\\
{\bf r31}         & Used for local variables or \emph{environment pointer}\\
{\bf f0}          & scratch\\
{\bf f1}          & parameter passing and return value\\
{\bf f2-f8}       & parameter passing\\
{\bf f9-13}       & scratch\\
{\bf f14-f31}     & Local variables\\
{\bf cr0-cr7}     & Conditional register fields, each 4-bit wide (cr0-cr1 and   cr5-cr7 are scratch)\\
{\bf lr}          & Link register (scratch)\\
{\bf ctr}         & Count register (scratch) \\
{\bf xer}         & Fixed-point exception register (scratch)\\
{\bf fpscr}       & Floating-point Status and Control Register\\

% {\bf v0-v1}         & scratch\\
% {\bf v2-v13}        & vector parameters\\
% {\bf v14-v19}       & scratch\\
% {\bf v20-v31}       & permanent\\
% {\bf lr}            & scratch, link-register\\
% {\bf ctr}           & scratch, count-register\\
% {\bf cr0-cr1}       & scratch\\
% {\bf cr2-cr4}       & permanent\\
% {\bf cr5-cr7}       & scratch\\
\hline
\end{tabular}
\caption{Register usage on System V ABI PowerPC Processor}
\end{table}

\paragraph{Parameter passing}

\begin{itemize}
\item Stack pointer (r1) is always 16-byte aligned. The EABI differs here - it is 8-byte alignment.
\item 8 general-purpose registers (r3-r10) for integer and pointer types.
\item 8 floating-pointer registers (f1-f8) for float (promoted to double) and double types.
\item Additional arguments are passed on the stack directly after the back-chain and saved return address (8 bytes structure) on the callers stack frame.
\item 64-bit integer data types are passed in general-purpose registers as a whole in two
 32-bit general purpose registers (an odd and an even e.g. r3 and r4), probably skipping an even integer register.
 or passed on the stack. They are never splitted into a register and stack part.
\item Ellipse calls set CR bit 6 

\end{itemize}

\paragraph{Return values}

\begin{itemize}
\item 32-bit integers use register r3, 64-bit use registers r3 and r4 (hiword in r3, loword in r4).
\item floating-point values are returned using register f1.
\end{itemize}

\pagebreak

\paragraph{Stack layout}

Stack frame is always 16-byte aligned. Stack directly after function prolog:\\

\begin{figure}[h]
\begin{tabular}{5|3|1 1}
\hhline{~-~~}
                                  & \vdots                     &                                &                              \\
\hhline{~=~~}
local data                        &                            &                                & \mrrbrace{6}{caller's frame} \\
\hhline{~-~~}
\mrlbrace{3}{parameter area}      & \ldots                     & \mrrbrace{3}{stack parameters} &                              \\
                                  & \ldots                     &                                &                              \\
                                  & \ldots                     &                                &                              \\
\hhline{~-~~}
                                  & saved return address (for callee) &                                &                              \\
\hhline{~-~~}
                                  & parent stack frame pointer &                                &                              \\
\hhline{~=~~}
local data                        &                            &                                & \mrrbrace{3}{current frame}  \\
\hhline{~-~~}
parameter area                    &                            &                                &                              \\
\hhline{~-~~}
                                  & \vdots                     &                                &                              \\
\hhline{~-~~}
\end{tabular}
\\
\\
\\
\caption{Stack layout on System V ABI for PowerPC 32-bit calling convention}
\end{figure}
\newpage
%//////////////////////////////////////////////////////////////////////////////
%
% Copyright (c) 2007,2009 Daniel Adler <dadler@uni-goettingen.de>, 
%                         Tassilo Philipp <tphilipp@potion-studios.com>
%
% Permission to use, copy, modify, and distribute this software for any
% purpose with or without fee is hereby granted, provided that the above
% copyright notice and this permission notice appear in all copies.
%
% THE SOFTWARE IS PROVIDED "AS IS" AND THE AUTHOR DISCLAIMS ALL WARRANTIES
% WITH REGARD TO THIS SOFTWARE INCLUDING ALL IMPLIED WARRANTIES OF
% MERCHANTABILITY AND FITNESS. IN NO EVENT SHALL THE AUTHOR BE LIABLE FOR
% ANY SPECIAL, DIRECT, INDIRECT, OR CONSEQUENTIAL DAMAGES OR ANY DAMAGES
% WHATSOEVER RESULTING FROM LOSS OF USE, DATA OR PROFITS, WHETHER IN AN
% ACTION OF CONTRACT, NEGLIGENCE OR OTHER TORTIOUS ACTION, ARISING OUT OF
% OR IN CONNECTION WITH THE USE OR PERFORMANCE OF THIS SOFTWARE.
%
%//////////////////////////////////////////////////////////////////////////////

% ==================================================
% PowerPC 64
% ==================================================
\subsection{PowerPC (64bit) Calling Convention}

\paragraph{Overview}

\begin{itemize}
\item Word size is 64 bits
\item Big endian (MSB) and litte endian (LSB) operating modes.
\item Apple Mac OS X/Darwin PPC is specified in "Mac OS X ABI Function Call Guide"\cite{ppcMacOSX}. It uses Big Endian (MSB).
\item Linux PPC 64-bit ABI is specified in "64-bit PowerPC ELF Application Binary Interface Supplement"\cite{ppcelf64abi} which is based on "System V ABI".
\end{itemize}


\paragraph{\product{dyncall} support}

\product{Dyncall} supports PowerPC (64bit) Big Endian and Little Endian ELF ABIs on System V systems (Linux, etc.). Mac OS X is not supported.


\subsubsection{PPC64 ELF ABI}

\paragraph{Registers and register usage}

@@@


\paragraph{Parameter passing}

@@@


\paragraph{Return values}

@@@


\paragraph{Stack layout}

@@@

\newpage
%
% Copyright (c) 2007,2010 Daniel Adler <dadler@uni-goettingen.de>, 
%                         Tassilo Philipp <tphilipp@potion-studios.com>
%
% Permission to use, copy, modify, and distribute this software for any
% purpose with or without fee is hereby granted, provided that the above
% copyright notice and this permission notice appear in all copies.
%
% THE SOFTWARE IS PROVIDED "AS IS" AND THE AUTHOR DISCLAIMS ALL WARRANTIES
% WITH REGARD TO THIS SOFTWARE INCLUDING ALL IMPLIED WARRANTIES OF
% MERCHANTABILITY AND FITNESS. IN NO EVENT SHALL THE AUTHOR BE LIABLE FOR
% ANY SPECIAL, DIRECT, INDIRECT, OR CONSEQUENTIAL DAMAGES OR ANY DAMAGES
% WHATSOEVER RESULTING FROM LOSS OF USE, DATA OR PROFITS, WHETHER IN AN
% ACTION OF CONTRACT, NEGLIGENCE OR OTHER TORTIOUS ACTION, ARISING OUT OF
% OR IN CONNECTION WITH THE USE OR PERFORMANCE OF THIS SOFTWARE.
%

% ==================================================
% ARM32
% ==================================================
\subsection{ARM32 Calling Convention}

\paragraph{Overview}

The ARM32 family of processors is based on 
the Advanced RISC Machines (ARM) processor architecture (32 bit RISC). 
The word size is 32 bits (and the programming model is LLP64).\\
Basically, this family of microprocessors can be run in 2 major modes:\\
\\
\begin{tabular}{2 B}
\hline
Mode          & Description\\
\hline
{\bf ARM}     & 32bit instruction set\\
{\bf THUMB}   & compressed instruction set using 16bit wide instruction encoding\\
\hline
\end{tabular}
\\
\\
For more details, take a look at the ARM-THUMB Procedure Call Standard (ATPCS) \cite{ATPCS}, the Procedure Call Standard for the ARM Architecture (AAPCS) \cite{AAPCS}, as well as the Debian ARM EABI port wiki \cite{armeabi}.


\paragraph{\product{dyncall} support}

Currently, the \product{dyncall} library supports the ARM and THUMB mode of the ARM32 family (ATPCS \cite{ATPCS} and EABI \cite{armeabi}), excluding manually triggered ARM-THUMB interworking calls. Although it's quite possible that the current implementation runs on other ARM processor families as well, please note that only the ARMv4t family has been thoroughly tested at the time of writing. Please report if the code runs on other ARM families, too.\\
It is important to note, that dyncall supports the ARM architecture calling convention variant {\bf with floating point hardware disabled} (meaning that the FPA and the VFP (scalar mode) procedure call standards are not supported).
This processor family features some instruction sets accelerating DSP and multimedia application like the ARM Jazelle Technology (direct Java bytecode execution, providing acceleration for some bytecodes while calling software code for others), etc. that are not supported by the dyncall library.\\


\subsubsection{ATPCS ARM mode}


\paragraph{Registers and register usage}

In ARM mode, the ARM32 processor has sixteen 32 bit general purpose registers, namely r0-15:\\
\\
\begin{table}[h]
\begin{tabular}{3 B}
\hline
Name         & Brief description\\
\hline
{\bf r0}     & parameter 0, scratch, return value\\
{\bf r1}     & parameter 1, scratch, return value\\
{\bf r2-r3}  & parameters 2 and 3, scratch\\
{\bf r4-r10} & permanent\\
{\bf r11}    & frame pointer, permanent\\
{\bf r12}    & scratch\\
{\bf r13}    & stack pointer, permanent\\
{\bf r14}    & link register, permanent\\
{\bf r15}    & program counter (note: due to pipeline, r15 points to 2 instructions ahead)\\
\hline
\end{tabular}
\caption{Register usage on arm32}
\end{table}

\paragraph{Parameter passing}

\begin{itemize}
\item stack parameter order: right-to-left
\item caller cleans up the stack
\item first four words are passed using r0-r3
\item subsequent parameters are pushed onto the stack (in right to left order, such that the stack pointer points to the first of the remaining parameters)
\item if the callee takes the address of one of the parameters and uses it to address other parameters (e.g. varargs) it has to copy - in its prolog - the first four words to a reserved stack area adjacent to the other parameters on the stack
\item parameters \textless=\ 32 bits are passed as 32 bit words
\item 64 bit parameters are passed as two 32 bit parts (even partly via the register and partly via the stack), although this doesn't seem to be specified in the ATPCS), with the loword coming first
\item structures and unions are passed by value, with the first four words of the parameters in r0-r3
\item if return value is a structure, a pointer pointing to the return value's space is passed in r0, the first parameter in r1, etc... (see {\bf return values})
\item keeping the stack eight-byte aligned can improve memory access performance and is required by LDRD and STRD on ARMv5TE processors which are part of the ARM32 family, so, in order to avoid problems one should always align the stack (tests have shown, that GCC does care about the alignment when using the ellipsis)
\end{itemize}

\paragraph{Return values}
\begin{itemize}
\item return values \textless=\ 32 bits use r0
\item 64 bit return values use r0 and r1
\item if return value is a structure, the caller allocates space for the return value on the stack in its frame and passes a pointer to it in r0
\end{itemize}

\paragraph{Stack layout}

Stack directly after function prolog:\\

\begin{figure}[h]
\begin{tabular}{5|3|1 1}
\hhline{~-~~}
                                         & \vdots &                                      &                              \\
\hhline{~=~~}
register save area                       &        &                                      & \mrrbrace{5}{caller's frame} \\
\hhline{~-~~}
local data                               &        &                                      &                              \\
\hhline{~-~~}
\mrlbrace{7}{parameter area}             & \ldots & \mrrbrace{3}{stack parameters}       &                              \\
                                         & \ldots &                                      &                              \\
                                         & \ldots &                                      &                              \\
\hhline{~=~~}
                                         & r3     & \mrrbrace{4}{spill area (if needed)} & \mrrbrace{7}{current frame}  \\
                                         & r2     &                                      &                              \\
                                         & r1     &                                      &                              \\
                                         & r0     &                                      &                              \\
\hhline{~-~~}
register save area (with return address) &        &                                      &                              \\
\hhline{~-~~}
local data                               &        &                                      &                              \\
\hhline{~-~~}
parameter area                           & \vdots &                                      &                              \\
\hhline{~-~~}
\end{tabular}
\caption{Stack layout on arm32}
\end{figure}


\newpage

\subsubsection{ATPCS THUMB mode}


\paragraph{Status}

\begin{itemize}
\item The ATPCS THUMB mode is untested.
\item Ellipse calls may not work.
\item C++ this calls do not work.
\end{itemize}

\paragraph{Registers and register usage}

In THUMB mode, the ARM32 processor family supports eight 32 bit general purpose registers r0-r7 and access to high order registers r8-r15:\\
\\
\begin{table}[h]
\begin{tabular}{3 B}
\hline
Name         & Brief description\\
\hline
{\bf r0}     & parameter 0, scratch, return value\\
{\bf r1}     & parameter 1, scratch, return value\\
{\bf r2-r3}  & parameters 2 and 3, scratch\\
{\bf r4-r6}  & permanent\\
{\bf r7}     & frame pointer, permanent\\
{\bf r8-r11} & permanent\\
{\bf r12}    & scratch\\
{\bf r13}    & stack pointer, permanent\\
{\bf r14}    & link register, permanent\\
{\bf r15}    & program counter (note: due to pipeline, r15 points to 2 instructions ahead)\\
\hline
\end{tabular}
\caption{Register usage on arm32 thumb mode}
\end{table}

\paragraph{Parameter passing}

\begin{itemize}
\item stack parameter order: right-to-left
\item caller cleans up the stack
\item first four words are passed using r0-r3
\item subsequent parameters are pushed onto the stack (in right to left order, such that the stack pointer points to the first of the remaining parameters)
\item if the callee takes the address of one of the parameters and uses it to address other parameters (e.g. varargs) it has to copy - in its prolog - the first four words to a reserved stack area adjacent to the other parameters on the stack
\item parameters \textless=\ 32 bits are passed as 32 bit words
\item 64 bit parameters are passed as two 32 bit parts (even partly via the register and partly via the stack), although this doesn't seem to be specified in the ATPCS), with the loword coming first
\item structures and unions are passed by value, with the first four words of the parameters in r0-r3
\item if return value is a structure, a pointer pointing to the return value's space is passed in r0, the first parameter in r1, etc. (see {\bf return values})
\item keeping the stack eight-byte aligned can improve memory access performance and is required by LDRD and STRD on ARMv5TE processors which are part of the ARM32 family, so, in order to avoid problems one should always align the stack (tests have shown, that GCC does care about the alignment when using the ellipsis)
\end{itemize}


\paragraph{Return values}
\begin{itemize}
\item return values \textless=\ 32 bits use r0
\item 64 bit return values use r0 and r1
\item if return value is a structure, the caller allocates space for the return value on the stack in its frame and passes a pointer to it in r0
\end{itemize}

\paragraph{Stack layout}

Stack directly after function prolog:\\

\begin{figure}[h]
\begin{tabular}{5|3|1 1}
\hhline{~-~~}
                                         & \vdots &                                      &                              \\
\hhline{~=~~}
register save area                       &        &                                      & \mrrbrace{5}{caller's frame} \\
\hhline{~-~~}
local data                               &        &                                      &                              \\
\hhline{~-~~}
\mrlbrace{7}{parameter area}             & \ldots & \mrrbrace{3}{stack parameters}       &                              \\
                                         & \ldots &                                      &                              \\
                                         & \ldots &                                      &                              \\
\hhline{~=~~}
                                         & r3     & \mrrbrace{4}{spill area (if needed)} & \mrrbrace{7}{current frame}  \\
                                         & r2     &                                      &                              \\
                                         & r1     &                                      &                              \\
                                         & r0     &                                      &                              \\
\hhline{~-~~}
register save area (with return address) &        &                                      &                              \\
\hhline{~-~~}
local data                               &        &                                      &                              \\
\hhline{~-~~}
parameter area                           & \vdots &                                      &                              \\
\hhline{~-~~}
\end{tabular}
\caption{Stack layout on arm32 thumb mode}
\end{figure}



\newpage

\subsubsection{EABI (ARM and THUMB mode)}


The ARM EABI is very similar to the ABI outlined in ARM-THUMB procedure call
standard (ATPCS) \cite{ATPCS} - however, the EABI requires the stack to be
8-byte aligned at function entries, as well as 64 bit parameters. The latter
are aligned on 8-byte boundaries on the stack and 2-registers for parameters
passed via register. In order to achieve such an alignment, a register might
have to be skipped for parameters passed via registers, or 4-bytes on the stack
for parameters passed via the stack. Refer to the Debian ARM EABI port wiki for more information \cite{armeabi}.


\paragraph{Status}

\begin{itemize}
\item The EABI THUMB mode is tested and works fine (contrary to the ATPCS).
\item Ellipse calls do not work.
\item C++ this calls do not work.
\end{itemize}

\newpage

\subsubsection{ARM on Apple's iOS (Darwin) Platform}


The iOS runs on ARMv6 (iOS 2.0) and ARMv7 (iOS 3.0) architectures.
Typically code is compiled in Thumb mode.

\paragraph{Register usage}

\begin{table}[h]
\begin{tabular}{3 B}
\hline
Name         & Brief description\\
\hline
{\bf R0}     & parameter 0, scratch, return value\\
{\bf R1}     & parameter 1, scratch, return value\\
{\bf R2-R3}  & parameters 2 and 3, scratch\\
{\bf R4-R6}  & permanent\\
{\bf R7}     & frame pointer, permanent\\
{\bf R8}     & permanent\\
{\bf R9}     & permanent(iOS 2.0) and scratch (since iOS 3.0)\\
{\bf R10-R11}& permanent\\
{\bf R12}    & scratch, intra-procedure scratch register (IP) used by dynamic linker\\
{\bf R13}    & stack pointer, permanent\\
{\bf R14}    & link register, permanent\\
{\bf R15}    & program counter (note: due to pipeline, r15 points to 2 instructions ahead)\\
{\bf CPSR}   & Program status register\\
{\bf D0-D7}  & scratch. aliases S0-S15, on ARMv7 alsa as Q0-Q3. Not accessible from Thumb mode on ARMv6.\\
{\bf D8-D15} & permanent, aliases S16-S31, on ARMv7 alsa as Q4-A7. Not accesible from Thumb mode on ARMv6.\\
{\bf D16-D31}& Only available in ARMv7, aliases Q8-Q15.\\
{\bf FPSCR}  & VFP status register.\\
\hline
\end{tabular}
\caption{Register usage on ARM Apple iOS}
\end{table}

The ABI is based on the AAPCS but with some important differences listed below:

\begin{itemize}
\item R7 instead of R11 is used as frame pointer
\item R9 is scratch since iOS 3.0, was preserved before.
\end{itemize}

\subsubsection{Architectures}

The ARM architecture family contains several revisions with capabilities and
extensions (such as thumb-interworking and more vector registers) 
The following table summaries important properties of the various 
architecture standards.

% iPhone 3GS : ARM Cortex-A8
% Nintendo DS: ARM 7 and ARM 9
% ARM 7: ARMv4T
% ARM 9: ARMv4T, HTC Wizard

\begin{table}[h]
\begin{tabular}{lll}
Arch   & Platforms & Details \\
\hline
ARMv4  & & \\
\hline
ARMv4T & ARM 7, ARM 9, Neo FreeRunner (OpenMoko) & \\
\hline
ARMv5  & & BLX instruction available \\
\hline
ARMv6  & & No vector registers available in thumb \\
\hline
ARMv7  & iPod touch, iPhone 3GS/4 & \\ 
\hline
\end{tabular}
\caption{Overview of ARM Architecture, Platforms and Details}
\end{table}

\newpage

\newpage
%
% Copyright (c) 2014,2015 Daniel Adler <dadler@uni-goettingen.de>, 
%                         Tassilo Philipp <tphilipp@potion-studios.com>
%
% Permission to use, copy, modify, and distribute this software for any
% purpose with or without fee is hereby granted, provided that the above
% copyright notice and this permission notice appear in all copies.
%
% THE SOFTWARE IS PROVIDED "AS IS" AND THE AUTHOR DISCLAIMS ALL WARRANTIES
% WITH REGARD TO THIS SOFTWARE INCLUDING ALL IMPLIED WARRANTIES OF
% MERCHANTABILITY AND FITNESS. IN NO EVENT SHALL THE AUTHOR BE LIABLE FOR
% ANY SPECIAL, DIRECT, INDIRECT, OR CONSEQUENTIAL DAMAGES OR ANY DAMAGES
% WHATSOEVER RESULTING FROM LOSS OF USE, DATA OR PROFITS, WHETHER IN AN
% ACTION OF CONTRACT, NEGLIGENCE OR OTHER TORTIOUS ACTION, ARISING OUT OF
% OR IN CONNECTION WITH THE USE OR PERFORMANCE OF THIS SOFTWARE.
%

% ==================================================
% ARM64
% ==================================================
\subsection{ARM64 Calling Convention}

\paragraph{Overview}

ARMv8 introduced the AArch64 calling convention. ARM64 chips can be run in 64 or 32bit mode, but not by the same process. Interworking is only intre-process.\\
The word size is defined to be 32 bits, a dword 64 bits. Note that this is due to historical reasons (terminology
didn't change from ARM32).\\
For more details, take a look at the Procedure Call Standard for the ARM 64-bit Architecture \cite{AAPCS64}.


\paragraph{\product{dyncall} support}

The \product{dyncall} library supports the ARM 64-bit AArch64 PCS ABI, for calls and callbacks.

\subsubsection{AAPCS64 Calling Convention}

\paragraph{Registers and register usage}

ARM64 features thirty-one 64 bit general purpose registers, namely x0-x30. Also, there is SP, a register with restricted use, used for the stack pointer, and PC dedicated as program counter. Additionally, there are thirty-two 128 bit registers v0-v31, to be used as SIMD and floating point registers, referred to as q0-q31, d0-d31 and s0-s31, respectively, depending on their use:\\

\begin{table}[h]
\begin{tabular}{3 B}
\hline
Name          & Brief description\\
\hline        
{\bf x0-x7}   & parameters, scratch, return value\\
{\bf x8}      & indirect result location pointer\\
{\bf x9-x15}  & scratch\\
{\bf x16}     & permanent in some cases, can have special function (IP0), see doc\\
{\bf x17}     & permanent in some cases, can have special function (IP1), see doc\\
{\bf x18}     & reserved as platform register, advised not to be used for handwritten, portable asm, see doc \\
{\bf x19-x28} & permanent\\
{\bf x29}     & permanent, frame pointer\\
{\bf x30}     & permanent, link register\\
{\bf SP}      & permanent, stack pointer\\
{\bf PC}      & program counter\\
\hline
\end{tabular}
\caption{Register usage on arm64}
\end{table}

\paragraph{Parameter passing}

\begin{itemize}
\item stack parameter order: right-to-left
\item caller cleans up the stack
\item first 8 integer arguments are passed using x0-x7
\item first 8 floating point arguments are passed using d0-d7
\item subsequent parameters are pushed onto the stack
\item if the callee takes the address of one of the parameters and uses it to address other parameters (e.g. varargs) it has to copy - in its prolog - the first 8 integer and 8 floating-point registers to a reserved stack area adjacent to the other parameters on the stack (only the unnamed parameters require saving, though)
\item structures and unions are passed by value, with the first four words of the parameters in r0-r3
\item if return value is a structure, a pointer pointing to the return value's space is passed in r0, the first parameter in r1, etc... (see {\bf return values})
\item stack is required to be throughout eight-byte aligned
\end{itemize}

\paragraph{Return values}
\begin{itemize}
\item integer return values use x0
\item floating-point return values use d0
\item otherwise, the caller allocates space, passes pointer to it to the callee through x8, and callee writes return value to this space
\end{itemize}

\paragraph{Stack layout}

Stack directly after function prolog:\\

\begin{figure}[h]
\begin{tabular}{5|3|1 1}
\hhline{~-~~}
                                   & \vdots &                                       &                              \\
\hhline{~=~~}                                                                       
register save area                 &        &                                       & \mrrbrace{5}{caller's frame} \\
\hhline{~-~~}                                                                       
local data                         &        &                                       &                              \\
\hhline{~-~~}                                                                       
\mrlbrace{13}{parameter area}      & \ldots & \mrrbrace{3}{stack parameters}        &                              \\
                                   & \ldots &                                       &                              \\
                                   & \ldots &                                       &                              \\
\hhline{~=~~}
                                   & x0     & \mrrbrace{10}{spill area (if needed)} & \mrrbrace{15}{current frame} \\
                                   & x1     &                                       &                              \\
                                   & \ldots &                                       &                              \\
                                   & x2     &                                       &                              \\
                                   & x7     &                                       &                              \\
                                   & d0     &                                       &                              \\
                                   & d1     &                                       &                              \\
                                   & \ldots &                                       &                              \\
                                   & d2     &                                       &                              \\
                                   & d7     &                                       &                              \\
\hhline{~-~~}                                                                       
register save area                 &        &                                       &                              \\
\hhline{~-~~}                                                                       
local data                         &        &                                       &                              \\
\hhline{~-~~}                                                                       
link and frame register            & x30    &                                       &                              \\
                                   & x29    &                                       &                              \\
\hhline{~-~~}                                                                       
parameter area                     & \vdots &                                       &                              \\
\hhline{~-~~}
\end{tabular}
\caption{Stack layout on arm64}
\end{figure}

\newpage

\newpage
%//////////////////////////////////////////////////////////////////////////////
%
% Copyright (c) 2007,2009 Daniel Adler <dadler@uni-goettingen.de>, 
%                         Tassilo Philipp <tphilipp@potion-studios.com>
%
% Permission to use, copy, modify, and distribute this software for any
% purpose with or without fee is hereby granted, provided that the above
% copyright notice and this permission notice appear in all copies.
%
% THE SOFTWARE IS PROVIDED "AS IS" AND THE AUTHOR DISCLAIMS ALL WARRANTIES
% WITH REGARD TO THIS SOFTWARE INCLUDING ALL IMPLIED WARRANTIES OF
% MERCHANTABILITY AND FITNESS. IN NO EVENT SHALL THE AUTHOR BE LIABLE FOR
% ANY SPECIAL, DIRECT, INDIRECT, OR CONSEQUENTIAL DAMAGES OR ANY DAMAGES
% WHATSOEVER RESULTING FROM LOSS OF USE, DATA OR PROFITS, WHETHER IN AN
% ACTION OF CONTRACT, NEGLIGENCE OR OTHER TORTIOUS ACTION, ARISING OUT OF
% OR IN CONNECTION WITH THE USE OR PERFORMANCE OF THIS SOFTWARE.
%
%//////////////////////////////////////////////////////////////////////////////

\subsection{MIPS Calling Convention}

\paragraph{Overview}

The MIPS family of processors is based on the MIPS processor architecture.
Multiple revisions of the MIPS Instruction set exist, namely MIPS I, MIPS II, MIPS III, MIPS IV, MIPS32 and MIPS64.
Today, MIPS32 and MIPS64 for 32-bit and 64-bit respectively.\\
Several add-on extensions exist for the MIPS family: 

\begin{description}
\item [MIPS-3D] simple floating-point SIMD instructions dedicated to common 3D tasks.
\item [MDMX] (MaDMaX) more extensive integer SIMD instruction set using 64 bit floating-point registers.
\item [MIPS16e] adds compression to the instruction stream to make programs take up less room (allegedly a response to the THUMB instruction set of the ARM architecture).
\item [MIPS MT] multithreading additions to the system similar to HyperThreading.
\end{description}

Unfortunately, there is actually no such thing as "The MIPS Calling Convention".  Many possible conventions are used
by many different environments such as \emph{32}, \emph{O64}, \emph{N32}, \emph{64} and \emph{EABI}.

\paragraph{\product{dyncall} support}

Currently, dyncall supports the EABI calling convention which is used on the Homebrew SDK for the Playstation Portable.
As documentation for this EABI is unofficial, this port is currently experimental.

\subsubsection{MIPS EABI 32-bit Calling Convention}

\paragraph{Register usage}

\begin{table}[h]
\begin{tabular}{lll}
\hline
Name                                   & Alias                     & Brief description\\
\hline                                                             
{\bf \$0}                              & {\bf \$zero}              & Hardware zero \\
{\bf \$1}                              & {\bf \$at}                & Assembler temporary \\
{\bf \$2-\$3}                          & {\bf \$v0-\$v1}           & Integer results \\
{\bf \$4-\$11}                         & {\bf \$a0-\$a7}           & Integer arguments\\
{\bf \$12-\$15,\$24,\$25}              & {\bf \$t4-\$t7,\$t8,\$t9} & Integer temporaries \\
{\bf \$25}                             & {\bf \$t9}                & Integer temporary, hold the address of the called function for all PIC calls (by convention) \\
{\bf \$16-\$23}                        & {\bf \$s0-\$s7}           & Preserved \\
{\bf \$26,\$27}                        & {\bf \$kt0,\$kt1}         & Reserved for kernel \\
{\bf \$28}                             & {\bf \$gp}                & Global pointer \\
{\bf \$29}                             & {\bf \$sp}                & Stack pointer \\
{\bf \$30}                             & {\bf \$s8}                & Frame pointer \\
{\bf \$31}                             & {\bf \$ra}                & Return address \\
{\bf hi, lo}                           &                           & Multiply/divide special registers \\
{\bf \$f0,\$f2}                        &                           & Float results \\
{\bf \$f1,\$f3,\$f4-\$f11,\$f20-\$f23} &                           & Float temporaries \\
{\bf \$f12-\$f19}                      &                           & Float arguments \\
\end{tabular}
\caption{Register usage on mips32 eabi calling convention}
\end{table}

\paragraph{Parameter passing}

\begin{itemize}
\item Stack parameter order: right-to-left
\item Caller cleans up the stack
\item Stack always aligned to 8 bytes.
\item first 8 integers and floats are passed independently in registers using \$a0-\$a7 and \$f12-\$f19, respectively.
\item if either integer or float registers are consumed up, the stack is used.
\item 64-bit floats and integers are passed on two integer registers starting at an even register number, probably skipping one odd register.
\item \$a0-\$a7 and \$f12-\$f19 are not required to be preserved.
\item results are returned in \$v0 (32-bit integer), \$v0 and \$v1 (64-bit integer/float), \$f0 (32 bit float) and \$f0 and \$f2 (2 $\times$ 32 bit float e.g. complex).
\end{itemize}

\paragraph{Stack layout}

Stack directly after function prolog:\\

\begin{figure}[h]
\begin{tabular}{5|3|1 1}
\hhline{~-~~}
                                  & \vdots                     &                                &                              \\
\hhline{~=~~}
register save area                &                            &                                & \mrrbrace{5}{caller's frame} \\
\hhline{~-~~}
local data                        &                            &                                &   \\
\hhline{~-~~}
\mrlbrace{3}{parameter area}      & \ldots                     & \mrrbrace{3}{stack parameters} &                              \\
                                  & \ldots                     &                                &                              \\
                                  & \ldots                     &                                &                              \\
\hhline{~=~~}
register save area (with return address) &                            &                                & \mrrbrace{5}{current frame}  \\
\hhline{~-~~}
local data                         &                            &                                &                              \\
\hhline{~-~~}
parameter area                    &                            &                                &                              \\
\hhline{~-~~}
                                  & \vdots                     &                                &                              \\
\hhline{~-~~}
\end{tabular}
\\
\\
\\
\caption{Stack layout on mips32 eabi calling convention}
\end{figure}

\newpage
%//////////////////////////////////////////////////////////////////////////////
%
% Copyright (c) 2012 Daniel Adler <dadler@uni-goettingen.de>, 
%                    Tassilo Philipp <tphilipp@potion-studios.com>
%
% Permission to use, copy, modify, and distribute this software for any
% purpose with or without fee is hereby granted, provided that the above
% copyright notice and this permission notice appear in all copies.
%
% THE SOFTWARE IS PROVIDED "AS IS" AND THE AUTHOR DISCLAIMS ALL WARRANTIES
% WITH REGARD TO THIS SOFTWARE INCLUDING ALL IMPLIED WARRANTIES OF
% MERCHANTABILITY AND FITNESS. IN NO EVENT SHALL THE AUTHOR BE LIABLE FOR
% ANY SPECIAL, DIRECT, INDIRECT, OR CONSEQUENTIAL DAMAGES OR ANY DAMAGES
% WHATSOEVER RESULTING FROM LOSS OF USE, DATA OR PROFITS, WHETHER IN AN
% ACTION OF CONTRACT, NEGLIGENCE OR OTHER TORTIOUS ACTION, ARISING OUT OF
% OR IN CONNECTION WITH THE USE OR PERFORMANCE OF THIS SOFTWARE.
%
%//////////////////////////////////////////////////////////////////////////////

\subsection{SPARC Calling Convention}

\paragraph{Overview}

The SPARC family of processors is based on the SPARC instruction set architecture, which comes in basically tree revisions,
V7, V8 and V9. The former two are 32-bit whereas the latter refers to the 64-bit SPARC architecture (see next chapter). SPARC is big endian.

\paragraph{\product{dyncall} support}

\product{dyncall} fully supports the SPARC 32-bit instruction set (V7 and V8), \product{dyncallback} support is missing, though.

\subsubsection{SPARC (32-bit) Calling Convention}

\paragraph{Register usage}

\begin{itemize}
\item 32 32-bit integer/pointer registers
\item 32 floating point registers (usable as 8 quad precision, 16 double precision or 32 single precision registers)
\item 32 registers are accessible at a time (8 are global ones (g*), whereas the rest forms a register window with 8 input (i*), 8 output (o*) and 8 local (l*) ones)
\item invoking a function shifts the register window, the old output registers become the new input registers (old local and input ones are not accessible anymore)
\end{itemize}

\begin{table}[h]
\begin{tabular}{lll}
\hline
Name                                 & Alias                   & Brief description\\
\hline
{\bf \%g0}                           &                         & Read-only, hardwired to 0 \\
{\bf \%g1-\%g7}                      &                         & Global \\
{\bf \%o0 and \%i0}                  &                         & Output and input argument 0, return value \\
{\bf \%o1-\%o5 and \%i1-\%i5}        &                         & Output and input argument registers \\
{\bf \%o6 and \%i6}                  &                         & Stack and frame pointer \\
{\bf \%o7 and \%i7}                  &                         & Return address (caller writes to o7, callee uses i7) \\
\end{tabular}
\caption{Register usage on sparc calling convention}
\end{table}

\paragraph{Parameter passing}
\begin{itemize}
\item Stack parameter order: right-to-left @@@ really?
\item Caller cleans up the stack @@@ really?
\item Stack always aligned to 8 bytes.
\item first 6 integers and floats are passed independently in registers using \%o0-\%o5
\item for every other argument the stack is used
\item @@@ what about floats, 64bit integers, etc.?
\item results are returned in \%i0, and structs/unions pass a pointer to them as a hidden stack parameter (see below)
\end{itemize}

\paragraph{Stack layout}

Stack directly after function prolog:\\

\begin{figure}[h]
\begin{tabular}{5|3|1 1}
\hhline{~-~~}
                                   & \vdots                      &                                &                               \\
\hhline{~=~~}
local data                         &                             &                                & \mrrbrace{10}{caller's frame} \\
\hhline{~-~~}
padding                            &                             &                                &                               \\
\hhline{~-~~}
\mrlbrace{7}{parameter area}       & argument x                  & \mrrbrace{3}{stack parameters} &                               \\
                                   & \ldots                      &                                &                               \\
                                   & argument 6                  &                                &                               \\
                                   & input argument 5 spill      & \mrrbrace{3}{spill area}       &                               \\
                                   & \ldots                      &                                &                               \\
                                   & input argument 0 spill      &                                &                               \\
                                   & struct/union return pointer &                                &                               \\
\hhline{~-~~}
register save area (\%i* and \%l*) &                             &                                &                               \\
\hhline{~=~~}
local data and padding             &                             &                                & \mrrbrace{3}{current frame}   \\
\hhline{~-~~}
parameter area                     &                             &                                &                               \\
\hhline{~-~~}
                                   & \vdots                      &                                &                               \\
\hhline{~-~~}
\end{tabular}
\\
\\
\\
\caption{Stack layout on sparc calling convention}
\end{figure}

\newpage
%//////////////////////////////////////////////////////////////////////////////
%
% Copyright (c) 2012 Daniel Adler <dadler@uni-goettingen.de>, 
%                    Tassilo Philipp <tphilipp@potion-studios.com>
%
% Permission to use, copy, modify, and distribute this software for any
% purpose with or without fee is hereby granted, provided that the above
% copyright notice and this permission notice appear in all copies.
%
% THE SOFTWARE IS PROVIDED "AS IS" AND THE AUTHOR DISCLAIMS ALL WARRANTIES
% WITH REGARD TO THIS SOFTWARE INCLUDING ALL IMPLIED WARRANTIES OF
% MERCHANTABILITY AND FITNESS. IN NO EVENT SHALL THE AUTHOR BE LIABLE FOR
% ANY SPECIAL, DIRECT, INDIRECT, OR CONSEQUENTIAL DAMAGES OR ANY DAMAGES
% WHATSOEVER RESULTING FROM LOSS OF USE, DATA OR PROFITS, WHETHER IN AN
% ACTION OF CONTRACT, NEGLIGENCE OR OTHER TORTIOUS ACTION, ARISING OUT OF
% OR IN CONNECTION WITH THE USE OR PERFORMANCE OF THIS SOFTWARE.
%
%//////////////////////////////////////////////////////////////////////////////

\subsection{SPARC64 Calling Convention}

\paragraph{Overview}

The SPARC family of processors is based on the SPARC instruction set architecture, which comes in basically tree revisions,
V7, V8 and V9. The former two are 32-bit (see previous chapter) whereas the latter refers to the 64-bit SPARC architecture. SPARC is big endian.

\paragraph{\product{dyncall} support}

\product{dyncall} fully supports the SPARC 64-bit instruction set (V9), \product{dyncallback} support is missing, though.

\subsubsection{SPARC (64-bit) Calling Convention}

@@@ finish


