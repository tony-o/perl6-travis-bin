%//////////////////////////////////////////////////////////////////////////////
%
% Copyright (c) 2007,2009 Daniel Adler <dadler@uni-goettingen.de>, 
%                         Tassilo Philipp <tphilipp@potion-studios.com>
%
% Permission to use, copy, modify, and distribute this software for any
% purpose with or without fee is hereby granted, provided that the above
% copyright notice and this permission notice appear in all copies.
%
% THE SOFTWARE IS PROVIDED "AS IS" AND THE AUTHOR DISCLAIMS ALL WARRANTIES
% WITH REGARD TO THIS SOFTWARE INCLUDING ALL IMPLIED WARRANTIES OF
% MERCHANTABILITY AND FITNESS. IN NO EVENT SHALL THE AUTHOR BE LIABLE FOR
% ANY SPECIAL, DIRECT, INDIRECT, OR CONSEQUENTIAL DAMAGES OR ANY DAMAGES
% WHATSOEVER RESULTING FROM LOSS OF USE, DATA OR PROFITS, WHETHER IN AN
% ACTION OF CONTRACT, NEGLIGENCE OR OTHER TORTIOUS ACTION, ARISING OUT OF
% OR IN CONNECTION WITH THE USE OR PERFORMANCE OF THIS SOFTWARE.
%
%//////////////////////////////////////////////////////////////////////////////

\newpage
\section{Building the library}

The library has been built and used successfully on several 
platform/architecture configurations and build systems.
Please see notes on specfic platforms to check if the target
architecture is currently supported.


\subsection{Requirements}

The following tools are supported directly to build the \product{dyncall} library.
However, as the number of source files to be compiled for a given
platform is small, it shouldn't be difficult to build it manually with
another toolchain.
\begin{itemize}
\item C compiler to build the \product{dyncall} library (GCC, Clang, SunPro or Microsoft C/C++ compiler)
\item C++ compiler to build the optional test cases (GCC, Clang, SunPro or Microsoft C/C++ compiler)
\item BSD make, GNU make, Microsoft nmake or mk (on Plan9) as automated build tools
\item Python (optional - for generation of some test cases)
\item Lua (optional - for generation of some test cases)
\item CMake (optional support)
\end{itemize}


\subsection{Supported/tested platforms and build systems}

Building \product{dyncall} is a straightforward two-step process, first configure, then make. The
library should be able to be built with the default operating systems' build tools, so BSD make on
BSD and derived systems, GNU make on GNU and compatible, mk on Plan9, nmake on Windows, etc..
This is a detailed overview of platforms and their build tools that are known to build
\product{dyncall}:\\
\\


\begin{tabular}{l l l}
{\bf{\large Platform}}             & Build Tool(s)    & Compiler, SDK                              \\
\hline
{\bf{\large Windows}}              & nmake,Visual Studio & cl, cygwin (gcc), mingw (gcc)              \\
{\bf{\large Unix-like}}            & GNU/BSD/Sun make & gcc, clang, sunc                           \\
{\bf{\large Plan9}}                & mk               & 8c                                         \\
{\bf{\large Haiku/BeOS}}           & GNU make         & gcc                                        \\
{\bf{\large iOS/iPhone}}           & GNU make         & gcc and iPhone SDK on Mac OS X             \\
{\bf{\large Nintendo DS}}          & nmake            & devkitPro\cite{devkitPro} tools on Windows \\
{\bf{\large Playstation Portable}} & GNU make         & psptoolchain\cite{psptoolchain} tools      \\

\end{tabular}\\

\pagebreak

\subsection{Build instructions}


\begin{enumerate}
\item Configure the source (not needed for Makefile.embedded)

\paragraph{*nix flavour}
\begin{lstlisting}
./configure [--option ...]
\end{lstlisting}

Available options (omit for defaults):

\begin{tabular}{ll}	
{\tt --help}                  & display help                             \\
{\tt --prefix={\it path}}     & specify installation prefix (Unix shell) \\
{\tt --target={\it platform}} & MacOSX,iOS,iPhoneSimulator,PSP,...       \\
{\tt --sdk={\it version}}     & SDK version                              \\
\end{tabular}

\paragraph{Windows flavour, and cross-build from Windows (PSP, NDS, etc.)}

\begin{lstlisting}
.\configure [/option ...]
\end{lstlisting}

Available options:

\begin{tabular}{ll}	
{\tt /?}                    & display help \\
{\tt /prefix {\it path}}    & set installation prefix (GNU make only) \\
{\tt /prefix-bd {\it path}} & set build directory prefix (GNU make only) \\
{\tt /target-x86}           & build for x86 architecture (default) \\
{\tt /target-x64}           & build for x64 architecture \\
{\tt /target-psp}           & build for PlayStation Portable (homebrew SDK) \\
{\tt /target-nds-arm}       & build for Nintendo DS (devkitPro, ARM mode) \\
{\tt /target-nds-thumb}     & build for Nintendo DS (devkitPro, THUMB mode) \\
{\tt /tool-msvc}            & use Microsoft Visual C++ compiler (default) \\
{\tt /tool-gcc}             & use GNU Compiler Collection \\
{\tt /asm-ml}               & use Microsoft Macro Assembler (default) \\
{\tt /asm-as}               & use the GNU Assembler \\
{\tt /asm-nasm}             & use NASM Assembler \\
{\tt /config-release}       & build release version (default) \\
{\tt /config-debug}         & build debug version \\
\end{tabular}


\paragraph{Plan 9 flavour}

\begin{lstlisting}
./configure.rc [--option ...]
\end{lstlisting}

Available options (none, at the moment):

\begin{tabular}{ll}	
{\tt --help} & display help \\
\end{tabular}


\item Build the static libraries \product{dyncall}, \product{dynload} and \product{dyncallback}
\begin{lstlisting}
make                      # for {GNU,BSD} Make
nmake /f Nmakefile        # for NMake on Windows
mk                        # for mk on Plan9
\end{lstlisting}

\item Install libraries and includes (supported for GNU and BSD make based builds, only)
\begin{lstlisting}
make install 
\end{lstlisting}

\item Optionally, build the test suite
\begin{lstlisting}
make tests                     # for {GNU,BSD} Make
nmake /f Nmakefile tests       # for NMake on Windows
mk tests                       # for mk on Plan9
\end{lstlisting}% @@@ check Plan9, should work already
\end{enumerate}

\subsection{Build-tool specific notes}

Some platforms require some manual tweaks:

Problem: Build fails because CC and/or related are not set, or different compiler,
linker, etc. should be used.

Solution: Set the 'CC' and other environment variables explicitly to the desired
tools. E.g.:

\begin{lstlisting}
CC=gcc make
\end{lstlisting}


Problem: On windows using mingw and msys/unixutils 'Make', the make uses
'cc' for C compilation, which does not exist in mingw.

Solution: Set the 'CC' environment variable explicitly to 'gcc' (as in
the example above).

\subsection{Build with CMake}

\begin{lstlisting}
cmake -DCMAKE_INSTALL_PREFIX=<location>
make
\end{lstlisting}

